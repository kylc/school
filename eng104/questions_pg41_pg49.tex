\documentclass[12pt,letterpaper]{article}

\usepackage{fullpage}

\pagestyle{empty}

\begin{document}

{\center\textbf{Optimists and Two Kinds} \\}

\begin{enumerate}

\item From the beginning, it seems that Ford is setting the marriage up for
failure.  First, we learn that Roy is not a very committed man; he works, but
takes whatever time off he can and calls it ``a workingman's paradise'' (Ford
33).  His mother ``was at home most of her time'' (33).  Boyd and his mother
were also in the perfect position for some kind of a relationship, spending
nights together playing canasta.  A relationship of some sorts would explain
Boyd's preconceived hate for Roy very nicely, this being especially evident when
Frank says ``He hardly knew him, but he did not like him.  I had no idea why.''
(35).  Also, the title of the story: Optimists.  Frank is being optimistic about
the lack of a relationship, even though all of the evidence points toward one
existing.  Then, at the end, we see that she has remarried.  She was not
devastated from the breakup or anything of that sort.  In fact, she seemed to
take it rather well.

However, it does not seem that she is in love with
Boyd.  That, or she was very good at covering it up.  She happily invites Roy to
play with Boyd and his wife, seemingly without hesitation: `` `Roy's home,' ...
`I hear Roy.  That's wonderful.' '' (34).  Conversely, it would explain Boyd's
preconceived hate for Roy.  This is especially evident when the main character
states that ``He hardly knew him, but he did not like him.  I had no idea why.''
(35).  Overall, it would seem that this case is ambiguous, however.  I think
that there is not enough evidence to prove that she was in love with Boyd, and
to keep such a lie going for so many years seems out of her character.

\item Ford certainly compresses that characters' speech during the heat of the
moment.  When Boyd is lying on the ground dying, the characters speak short
simple sentences and Ford places long descriptive sentences between each dialog.
However, later in the story when he meets his mother, when Spivey ``didn't seem
in any hurry'' (40), the characters speak freely to each other with descriptive
sentences only between every few pieces of conversation.  By providing less
dialog during intense moments, Ford allows the reader to fully capture the
emotion of that one sentence, or clip.  Rather than diluting the speech with
extra details, he captures the moment in a single dialog.  However, once
emotions are less intense and the characters are reminiscing, they speak in more
open dialog to relay what the character is thinking, not what Ford thinks is
important.

\item After so many years, I think it would be impossible for him to remember
the exact words spoken.  We must accept that Ford has a bit of dramatic leeway
in his writing, cutting out superfluous dialog and focusing on the pivotal
points.  Really, though, the exact words aren't important.  In fact, the exact
words could very well be quite boring, dragging the story down with on-the-spot
dialog instead of that crafted by a skilled writer.

\item Ford includes Dick Spivey perhaps to show that Frank's mother has moved
on.  For all we know from Frank's perspective, she could have disappeared
somewhere or gone totally insane.  Showing that she is living in what seems to
be a stable relationship reinforces our idea that she has moved on from the
events of the past, and that they no longer haunt her.  Spivey does not speak
because anything he says would be totally unrelated.  The important fact is his
existence and the relationship, not who he is or what he is like.  Had Spivey
been left out, we would have little context as to how his mother was faring.
Was she still in love with Roy, and constantly seeking him out?  Had she been so
devastated by the breakup of their marriage that she wasn't even trying anymore?
These questions would not have been answered.

\item Frank is an optimist, sometimes.  He is totally optimistic about the lack
of a relationship between his mother and Boyd, despite all of the evidence.
However, his overall outlook on life seems very dreary.  For instance, he states
that ``The most important things of your life can change so suddenly, so
unrecoverably, that you can forget even the most important of them an their
connections, you are so taken up by the chanciness of all that's happened and by
all that could and will happen next.'' (39).  This quote speaks of that past,
where it would seem as though he was an optimist.  Now, however, he is taking a
more pessimistic style simply by stating the above.  He is acknowledging that he
was too optimistic, and was swept up by the hopes of opportunities ahead.

\end{enumerate}

\begin{enumerate}

\item The daughter has to categories of responses: at first, she is excited
about the though of becoming a prodigy.  In fact, she is ``just as excited as
[her] mother'' (Tan 43).  As the story develops and she starts to believe that
she isn't good at anything, she develops a lazier response.  She no longer
believes that she can be the type of prodigy her mother wants her to be.  She
seems more content on annoying her mother than even attempting to please her.
The main shift in her mind seems to come very early.  Once her mother's tests
became routine, and once she realized that she was not doing well, she
essentially gave up.

\item Her mother's dialect makes her seem uneducated, not only in English but in
Chinese as well.  This reflects back on her hopes of her daughter becoming a
prodigy: she wants her daughter to be what she wasn't.  The mother also says
that the daughter is ungrateful.  This could refer to the fact that the mother
grew up in an environment where she didn't have opportunities like the daughter
has now.  In her mind, the daughter is taking all of these chances to become
something greater for granted, and mostly ignoring them.

\item  The narrator certainly references some factual information that would
suggest the author is a first-generation American.  For instance, she says that
she only knows Sacramento because that ``was the name of the street we lived on
in Chinatown'' (44).  Also, she seems to at least know Chinese, making a
reference to ``Ni kan'' which, apparently, means ``Look here.''

\item The narrator speaks in a very natural English dialect.  Her sentences flow
properly and sound very refined.  The mother, on the other hand, speaks in very
broken English, something that would be common of a first-generation immigrant.

\item In writing ``Two Kinds'', Tan reflects on her actions as somewhat foolish
and ungrateful.  She portrays herself in a bad light, even though she could have
done the exact opposite (being the storyteller).  Therefore, I believe that she
does find truth in the statement ``you could be anything you wanted to be in
America.''  Tan also states that ``I realized as a young reader, I could really
go anywhere.  I could go the prairie and the big woods, I could be living in a
different time.'' (42).

\end{enumerate}
\end{document}
