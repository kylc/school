\documentclass[11pt,letterpaper]{article}

\usepackage{ifpdf}
\usepackage{mla}
\usepackage[utf8]{inputenc}

\begin{document}
\begin{mla}{Kyle}{Cesare}{Davison}{ENG104}{November 29, 2011}{Feminism, by Means
of Insanity}

% -- SECTION 1 --
% Introduction (1-1.5 pages)
% - Background material on the author
%   - Biographical aspect
%     - Writer, feminist, social critic
%     - Lived during civil war, reconstruction, feminism (19th Amendment)
%     - Fell to depression, underwent rest cure, recovered by ignoring doctor
%   - Works and themes
%     - Women and Economics (1898) - nonfiction treatise on women's rights
%   - Historical, political, biographical contexts of the story`
%     - Early Women's Rights Movement (Elizabeth Cady Stanton, Susan B. Anthony)
% - Theme statement
%   - Strifes of feminism
%   - Gender roles, in marriage
Charlotte Perkins Gilman was a late 19th century and early 20th century American
writer and staunch supporter of the first-wave feminist movement (Delbanco and
Cheuse 220).  She lived during an exciting time in American history, seeing the
effects of the Civil War and Reconstruction, and seeing the progression of the
Feminist Movement from its infancy all the way up to the Nineteenth Amendment,
which gave women the right to vote.  These times were reflected in the themes of
Gilman's writings, including a book titled \emph{Women and Economics} which
argued in favor of women's rights to financial independence, and attempted to
convince women to change their cultural identities (DiGrazia).  Tumultuous times
struck Gilman in the early 1880s.  After giving birth to her first child,
Katharine, she came down with a serious bout of post-partum depression.  Being
the 19th century, such medical knowledge was still primitive.  The leading
medical figure at the time was Dr. Weir Mitchell, a primary proponent of the
``rest cure'' as it was called.  This rest cure called for minimal physical and
mental activity, theoretically allowing the body to concentrate only on healing
itself.  The rest cure also prescribed a ban on all artistic pursuits as they
were deemed too strenuous.  This included writing, a very large part of Gilman's
life at the time.  She was prescribed these instructions, but, over the next
weeks, her condition only worsened.  Only when she became active again, started
travelling, and, perhaps most importantly, writing, did she begin to recover
from her ``sickness.'' (``About Charlotte Perkins Gilman'')

In 1890, she wrote \emph{The Yellow Wallpaper}, a short story that promotes the
feminist cause of the 19th century and women rising above their societal gender
roles, and attempts to disprove psychological and physiological beliefs of women
at the time; the author makes this apparent by showing us the inner workings of
a woman's mind as she is led by society into insanity.  The story was largely
influenced by her bout with depression and her poor experiences with Mitchell's
rest cure, of which she makes reference to in the story (``About Charlotte
Perkins Gilman'').

% -- SECTION 2 --
% Analysis
% - What is the story about?
% - Significance of the wallpaper
%   - Plot structure (exposition, rising action, conflict, climax, peripety,
%       anagnorisis, falling action, resolution, and denouement)
%   - Narrative style
%   - Point-of-view
%   - Symbolism
%   - Repetition
%   - Juxtaposition
%   - Dialogue, monologue
%   - Character analysis
%   - Additional irony
On its surface, \emph{The Yellow Wallpaper} tells the story of a woman
descending into madness.  The main character, Jane, is suffering from some sort
of ailment; whether it is physical or mental, we aren't yet sure.  Her husband,
John, is a ``physician of high standing'' (Gilman 221) and, backed up by Jane's
brother, also a physician, has played her illness down as a simple ``temporary
nervous depression---a slight hysterical tendency'' (Gilman 221).  These
physicians, practicing only what was known in the time period, prescribed the
rest cure, in which she is ``absolutely forbidden to `work' until [she is] well
again.'' (Gilman 221)  Their marriage is not a bad one, but it is based on
inequality.  Jane even mentions that John is ``very careful and loving.''
(Gilman 222)  Jane is content with listening to John because, as it was thought,
he had a more complex brain and was able to stand back and look at a topic
rationally---something that a woman couldn't do.  John has settled on the idea
that Jane is sick in body, but Jane has come to the conclusion that she is sick
in mind.  John believes, essentially, that her sickness is that she is a woman.
This division separates them when Jane states that she is ``Better in body
perhaps,'' (Gilman 225) hinting that she believes she is still sick in the mind.
Her husband strongly disagrees, and begs her to ``never for one instant let that
idea enter [her] mind!'' (Gilman 225)

In hopes of Jane's conditions improving, they move to a summer home---``a
colonial mansion, a hereditary estate.'' (Gilman 221)  The room she is staying
in, which used to be a nursery, is at the very top of the house.  It is ``a very
airy room,'' (Gilman 222) but ``the windows are barred for little children.''
(Gilman 222)  This is an early symbol of women's repression in society, and how
they are treated as children and incapable of having thoughts of their own.

Soon after moving into their summer home, we are introduced to the yellow
wallpaper.  In her introducion of the wallpaper, we learn a great deal about her
thoughts in a very symbolic style.  She states that the wallpaper ``[commits]
every artistic sin,'' (Gilman 222) just as she is committing artistic sin in
writing while she is supposed to be resting.  In a bit of foreshadowing, she
also says that the stripes ``suddenly commit suicide---plungle off at outrageous
angles, destroy themselves.'' (Gilman 222)  The allusion here becomes more
apparent later in the story, when she contemplates ``[jumping] out of the
window.'' (Gilman 228)  This is stage one of her insanity: projection.  She is
projecting her depression onto an inanimate, unfeeling, indifferent object.  The
language helps to forward that cause: we know that wallpaper cannot commit
suicide, but we know that Jane can.

The second stage of her insanity is obsession.  She becomes totally immersed in
the wallpaper, examining it and forcing herself to think that she ``\emph{will}
follow that pointless pattern to some sort of conclusion.'' (Gilman 224)  She
even admits to herself in that very sentence that the pattern is pointless, yet
she still searches out some sort of factual meaning in it.  She meticulously
combs over every aspect of the design, trying to conclude \emph{something}, but
achieving nothing.  She sits ``there on this great immovable bed... and follow
that pattern about by the hour.'' (Gilman 224)  She even admits to herself that
she is ``getting really fond of the room in spite of the wallpaper.  Perhaps
\emph{because} of the wallpaper.'' (Gilman 224)  Such a contradictory statement
can only be attributed to some form of insanity.

Next, her insanity becomes totally outwardly apparent.  She says that ``the
woman behind [the wallpaper] is as plain as can be'' (Gilman 226).  This is a
shift from her previous metaphorical position, in which she states that ``it is
like a woman'' (Gilman 225).  Now, rather than simply viewing the figure and
determining that it is somewhat in the shape of a woman, she is totally
enthralled by the wallpaper and believes that there is a real woman inside of
it.  By this stage, she had begun creeping, or crawling, around the walls of the
room.  She doesn't realize what she had been doing, but she notices ``a streak
that runs round the room.  It goes behind every piece of furniture, except the
bed, a long, straight, even smooch, as if it had been rubbed over and over.''
(Gilman 226)  The rubbing is from her shoulder, which is why she has ``yellow
smooches on all [her] clothes.'' (Gilman 226)  This creeping is symbolic of how
little freedom she, and all women, had in society at the time.  She felt so
repressed by her husband, by physicians, and trying to fit in with typical
gender roles, that she believed that being able to creep around the room was
real freedom.  This repressiveness is emphasized again later on, when she
contemplates suicide but decides against it on the grounds that it's ``improper
and might be misconstrued,'' (Gilman 228) a relatively insignificant problem
when compared with the weight of the decision at hand.

% TODO: Combine
She imagines that everyone is out to get her.  She believes that the sister is
trying to trick her with every conversation, but commends herself in warding off
such cons: ``How she betrayed herself that time!'' (Gilman 227)  She continues,
``Jennie wanted to sleep with me---the sly thing.'' (Gilman 227)  She even
portrays John, the previously loving and caring husband, as being against her:
``John knows I don't sleep very well at night, for all I'm so quiet!  He asked
me all sorts of questions too, and pretended to be very loving and kind.  As if
I couldn't see through him!'' (Gilman 227)  Even though it appears to you and me
that John and the sister are acting in a totally benevolent way, the narrator
sees them as trying to keep her away from the wallpaper and her freedom.

Finally, her insanity reaches its peak.  She beings to experience a small amount
of synesthesia, a neurological condition in which the senses are mixed up.  She
notices that the smell of the wallpaper seems to follow her everywhere, and that
that the smell is ``like the \emph{color} of the paper!'' (Gilman 226) With only
a few days left before they must leave the house, Jane begins to have visions of
the woman trapped inside the wallpaper.  She sees ``her on that long road under
the trees, creeping along, and when a carriage comes she hides under the
blackberry vines.'' (Gilman 227) She decides that she wants to free the trapped
woman, and so she decides to peel off the wallpaper and set her free.  With only
one day remaining, she manages to get the wallpaper off.  She begins to believe
that she is the woman in the wallpaper, ``[wondering] if they all come out of
the wallpaper as I did!'' (Gilman 228).  She tells John that she ``got out at
last...  in spite of you and Jane,'' (Gilman 228) referring to herself in third
person.  She is creeping around the room when John enters, and John promptly
faints upon seeing her in such a condition.  Jane merely continues to creep,
crawling over John's body on each rotation around the room.  This could be
considered symbolic of a woman's victory over a repressive society, as she is
now unconstrained by her husband and is able to, literally, tread all over him.
John's fainting also pokes holes in the scientific knowledge of the time.  It
was thought that men were incapable of fainting, for it was a symptom of
hysteria, something that only affected women.  This is Gilman's way of, once
again, pushing a feminist theme by showing greater equality between the genders.

Given Jane's insanity, Gilman also presents us with a figure that represents the
ideal woman---one who falls within society's standard.  John's sister, Jennie,
``is a perfect and enthusiastic housekeeper, and hopes for no better
profession.'' (Gilman 223)  These are Jane's own words; Jennie is perfectly
content with being a housekeeper, just as a typical woman should be.  Jennie is
always portrayed in very womanly roles, whether it's ``getting the house ready''
(Gilman 225) while John and Jane are planning a trip, or watching Jane while
John is away.

Of course, the above reading is not the only possible interpretation of the
story.  One category into which \emph{The Yellow Wallpaper} is often placed is
that of 19th century Gothic horror (Pazhavila).  This category of literature is
classified by its ``stereotypical ominous mansion'' (Pazhavila) coupled with
``romance, abduction, \emph{insanity} [emphasis added], murder, doppelgangers,
and supernatural apparitions.''  \emph{The Yellow Wallpaper} easily fits these
categories, featuring the mansion, insanity, and supernatural apparitions
manifested in the wallpaper (Pazhavila).  Katherine Quinsey notes that Gothic
horror pieces focusing on women often portrayed the woman as trying to be
considered on equal terms as the men, just as Jane is fighting for the freedoms
John already has.

% -- SECTION 3 --
% Conclusion
In conclusion, Gilman's \emph{The Yellow Wallpaper} is a short story that
promotes the feminist cause in the 19th century and women rising above their
typical gender roles, and argues against common psychological and physiological
beliefs at the time, including Dr. Weir Mitchell's ``rest cure.''  It guides us
through this theme by telling the story of a woman going insane by a combination
of the rest cure and societal repression as a woman.  The main character is
driven to total insanity, resulting in a sort of pseudo-freedom only acceptable
to someone who has completely lost their mind.

\begin{workscited}

% TODO: Alphabetize these

\bibent ``About Charlotte Perkins Gilman.'' \emph{Charlotte Perkins Gilman
Society}. Ed.  Heidi Silcox. N.p., n.d. Web. 16 Nov. 2011. 

\bibent Delbanco, Nicholas, and Alan Cheuse, eds. \emph{The Yellow Wallpaper}.
\emph{The Yellow Wallpaper}. Ed. Charlotte P. Gilman. New York: McGraw-Hill,
2010. 220.  Print. 

\bibent DiGrazia, Jodi. ``Charlotte Perkins Gilman 1860-1935.'' The College of
Staten Island of CUNY, 14 Dec. 1998. Web. 14 Nov. 2011. 

\bibent Gilman, Charlotte, \emph{The Yellow Wallpaper}.  \emph{The Yellow
Wallpaper}. Literature Craft \& Voice. Ed. Nicholas Delbanco, and Alan Cheuse.
New York: McGraw-Hill, 2010. 221-28.  Print. 

\bibent Pazhavila, Angie. ``The Female Gothic Subtext: Gender Politics in
Charlotte Bronte's Jane Eyre and Charlotte Perkins Gilman's The Yellow
Wallpaper.'' \emph{Lethbridge Undergraduate Research Journal} 1.2 (2007). Web.
27 Nov. 2011. 

\end{workscited}
\end{mla}
\end{document}
