\documentclass[12pt,letterpaper]{article}

\usepackage{fullpage}

\pagestyle{empty}

\begin{document}

{\center\textbf{A\&P and The Story of An Hour} \\}

\begin{enumerate}

\item First, the three girls walk in ``in nothing but bathing suits'' (Update
5).  The narrator emphasizes the girls' clothing, which sets a precedent for the
rest of the story.  He is momentarily distracted by a perturbed customer, but
quickly refocuses his attention on the girls.  He follows them as they wonder
about the store, going from aisle to aisle and asking for help.  The main
character lets his imagination wander, recalling those less beautiful women who
often make appearances in his store.  He is extracted from his imagination when
the girls round the aisle, ready to check out.  Lengel, the store owner, also
shows up, and gets angry at the girls for their clothing, or lack thereof.
After ringing them up, the main character declares ``I quit'' (7).  This is the
final decision that the rest of the story had been building up to.

So, what drove him to make such a startling decision?  There are a few clues the
author drops around the story, but the main reason is that he wants to impress
the girls; he wants to be their hero, going against what society says he should
be doing.

\item If the author began by describing the beginning of Sammy's entire day
rather than just his work in the store, the story would lose a significant
portion of its meaning.  The beginning would add very little value, as the final
decision has everything to do with what occurs in the store, and nothing to do
with what occurs anywhere else.  Adding such superfluous details would simply
dilute the story.

The plot could also be dramatically different.  After all, we would be seeing an
entirely new part of his life.  We would lose that sense of monotony that Updike
manages to portray of working in the grocery store.  We would see that he lives
a more complete life, and might begin to think that his decision at the end was
somewhat foolish.

\item Sammy tells the story from the point of view of a tired, bored cashier.
He talks about how most of the women who enter the store ``with six children and
varicose veins mapping their legs and nobody, including them, could care less.''
(7).  Essentially, an appearance of such attractive girls is not a typical
occurrence for him, so he introduces some bias into the story.

If the story was told from the point of view of Stokesie, the girls would
probably have taken on a more negative role.  Stokesie is ``married, with two
babies chalked up on his fuselage already'' (6).  He expresses his disapproval
of the girls' entrance to Sammy, asking ``Is it done?'' (6).

Had the story been told by Queenie, we would have missed out on the entire
importance.  She misses the most important part of the story: Sammy's quiting.
Without this important event, the story would seem to be about nothing, simply a
stream of events with little significance.

\item Updike absolutely painted a very clear image of exactly what Sammy was
seeing.  Hearing the story from Sammy's viewpoint aided in this feat immensely,
because we didn't necessarily know or care what anyone else was thinking or
doing.  We were totally immersed in Sammy's feelings, and he was totally
immersed in the girls.

The first important image I recognized was the description of the annoying
customer.  Updike describes how pleasant the girls are to look at as they make
their entrance, but the train of though is interrupted by a ``one of those cash
-register watchers'' (5).  Sammy is angered by her, and the reader is sucked up
into his feelings.  Sammy antagonizes her by calling her someone who had her day
made simply by seeing someone mess up.

Another interesting image is that of the cash register.  Updike describes all of
the sounds it makes, and the rhythm is seems to follow.  This is, in a way,
foreshadowing what his quitting.  He describes the monotony of working at the
store with a lame song.

\item Sammy quits his job for a pretty basic reason: he wants to be a hero for
the girls; to stand up against oppression.  He, arguably, succeeds in the
seconds, but the girls fail to take notice of him.  He seems to reach for a
moment of insight at the end, realizing that, if he is going to go against the
grain of society, he's going to get burned.  This admittance would imply that he
intends to continue this newfound trend, even though it may be difficult.

\item The most obvious alternative ending is perhaps even more ironic.  Had the
girls actually noticed him, it seems doubtful that they would see him as their
``hero'' (7).  They probably wouldn't even understand why he was quitting.  This
would probably put Sammy in an even worse emotional state.  No longer can he use
the excuse that they simply didn't hear him to qualify his not being a hero.
They did hear him, and they didn't really care.  Given his character, I think he
probably would have still come to the ending conclusion of ``how hard the world
was going to be from here on in'' (8).

\end{enumerate}

\begin{enumerate}

\item In the beginning, Mrs. Mallard has a neutral set of emotions.  Once she
learns of the death of her husband, her immediate response is one of sadness.
She reflects that he had loved her, and her him.  However, she begins to peer
out the wide, open window and realizes what her life could become.  Her emotions
move upward, thinking of her new freedom from oppression.  Finally, when she
comes back down and sees her husband enter the front door, her emotions take an
immediate nosedive and she dies of a heart attack.

Even though the story says that she died ``of joy that kills'' (Chopin 14), she
dies of disappointment.  The story is referring to her earlier joy, and, had it
been absent, she wouldn't have been shocked so intensely by the news that her
husband was alive.  So, in this case, the literal meaning of the story is the
exact opposite of the actual meaning.

\item I don't think that Chopin does a very good job of providing visual
imagery, but I don't think she was trying to do so.  The majority of the
information we get is in the form of emotions.  We follow Mrs. Mallard's
emotions very closely, even though we never hear much of her appearance or of
the appearance of the setting.

\item Chopin seems to take a generally positive view on Louise's behavior.  The
name ``Mrs. Mallard'' could imply multiple interpretations, but I think it
foreshadows her freedom.  Rather than living as a cooped up wife, she will move
on to take on the male's role, being free to do whatever she likes.

\item Through Mrs. Mallard's thoughts, we learn that she lived in a very
oppressive marriage, even though her husband loved her.  Apparently, her husband
imposed his ``private will upon a fellow-creature'' (14).  However, ``she had
loved him'' (14), though her dialog is not terribly convincing.  It sounds as
though she thought that she loved him, but in reality it was just more of his
manipulation.  Without more background information, it is hard to tell if this
is really the case.

\item I think that Chopin reinforces the significance of Louise's death by
announcing her heart condition at the beginning of the story.  The reader
initially expects her to suffer a heart issue upon hearing the news that her
husband has died.  Instead, she is very happy.  In an ironic twist, she actually
suffers heart failure upon learning that her husband is, in fact, alive.

\end{enumerate}
\end{document}
