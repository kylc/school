\documentclass[12pt,letterpaper]{article}

\usepackage{fullpage}

\pagestyle{empty}

\begin{document}

\begin{flushright}
Kyle Cesare
\end{flushright}

{\center\textbf{Greasy Lake and Araby} \\}

\begin{enumerate}

\item The climax of the story is in paragraph 9, in which the fight starts and
``The first lusty Rockette kick of his steel-toed boot caught me under the
chin'' (Boyle 78).  Beyond that, the story is mostly explaining their escape and
hiding, as they watch the men destroy their car.  There is another exciting
event at the end when they run into the girls, but it's just a small blip
compared to the earlier events.

The events don't really lead to a resolution.  Sure, the characters escape from
``this bad greasy character'' (Boyle 78), but they never meet any of the
consequences they talked about.  He never had to think of an excuse for his
parents, as he describes in paragraph 31.

\item The narrator runs into ``Something unspeakable, obscene, something soft,
wet, moss-grown'' (Boyle 79).  He realized that ``it gave like flesh'' (79) and
moves to the conclusion that he had run into a dead body floating out in the
lake.  He is primarily repulsed by such a terrifying encounter, and goes back to
the root of the problem: dropping the keys.  He wonder what would have happened
had he not made such a stupid mistake.  He never would have gotten in the fight,
he never would have tried to rape the girl, and he wouldn't be swimming through
this lake in which he finds a body bobbing around.
% TODO: How does this differ from the other ones?

\item Boyle mentions that ``Digby wore a gold star in his right ear and allowed
his father to pay his tuition at Cornell; Jeff was thinking of quitting
school\dots'', giving us a good hint of their social status.  Cornell is by no
means an inexpensive school to attend, so Digby's family must be quite well off.
He also mentions that the narrator is driving a Bel Air.  This is a pretty
middle-class car.  It also seems that production of this car ended around the
1960s, and the band the author mentions (the Toots \& the Maytals) were popular
around the 1960s.  Therefore, the story was probably set in the 1960s.  They
also say that Tony drove a '57 Chevy, so we can conclude that the group is
pretty well off to be able to afford cars and college tuitions.

\item The narrator certainly seems to want to give the outward appearance that
he is ``bad'', bud he throws in some odd language that seems to contradict
whether or not the actual sentiment is there.  For instance, he says that he and
his friends drink a bunch of ``gin... Tango, Thunderbird, and Bali Hai'' (Boyle
77), but also ``grape juice'' (77).  He also says that they read Andr\'{e} Gide,
a man known more for his intellectual honesty and strict adherence to one's
values.  These are certainly contradictory, and make it seem like he doesn't
really see himself as ``bad'', even if he projects such an image.

\item Had the story been written in present tense, we would have missed out on
the details that the narrator realized only after the event.  For instance, when
being attacking by the ``fox'', he says that ``I think it was the toenails that
did it.  Sure, the gin and the cannabis and even the Kentucky Fried may have had
a hand in it'' (Boyle 78).  Without time to inspect this later, after
experiencing the attack, he would not have known this, and the story would not
have included it.  The writing would have been more stream-of-conscious, without
as much detail and introspection into the events.

\item The boys have their ``tails between their legs'' now because they had just
been knocked down a peg.  They thought they were ``bad'', but they backed down
from that position when they ran into some people that they though really were
``bad.''  The narrator even says that he ``wanted to go home to [his] parents'
house and crawl into bed.'' (Boyle 81).

\end{enumerate}

\begin{enumerate}

\item The plot is launched at the beginning of paragraph 6, because he recalls
``One evening`` (Joyce 87) instead of reciting general information.

\item The sister can't go to Araby, so the narrator decides to go and bring
something back for her.  He wanted to go to Araby, so all important work in his
life fell away.  His uncle arrived home late, so he was late to the bazaar.  He
was late to the bazaar, so he wasn't able to get what he wanted.  He couldn't
buy something for the girl, so he essentially gave up all hope.

\item The isolated dialog does advance the plot, especially between the narrator
and the sister.  It reflects their true conversations, only short, sporadic
exchanges of words with little meaning.  Another example is when he is talking
to his uncle.  He doesn't really care what his uncle has to say, because it is
just a delay, stalling his attempts to leave for the bazaar.  Therefore, as he
isn't really listening, he doesn't recite the dialog to the reader because it
really doesn't matter.

\item Simply based on the narrator's extensive recall of the subject, down to
the dialog spoken, we can assume that it was not too long ago.  We do know that
he was quite young when the events occurred because he was still in school, his
father calls him by ``Yes, boy'' (Joyce 88), and he needed to get money from his
father before he could go to the bazaar.  As far as social status, he seems to
be pretty well off.  He doesn't attempt to be frugal, not being able to find a
sixpenny, he ``[handed] a shilling to a weary-looking man.'' (88).

\item The dead priest reflects the narrator's future aspirations.  It is ironic
because the priest is dead before the story even begins, and only at the end of
the story does the narrator realize that he will never be anything more to the
sister.  After all, he just gives up, unable to even buy a simple gift.

\item When the narrator expresses that he is driven by vanity, he is realizing
that he has always been that way.  We can see this because the narration,
relayed some time after the events, does not seem to hold contempt for his
actions at that time.  Of course he is angry about it, ``[his] eyes burned with
anguish and anger'' (Joyce 89), but that does not make it any less so.  If he
had changed right then, after realizing his faults, the narration would have
been in a different style.

\end{enumerate}
\end{document}
