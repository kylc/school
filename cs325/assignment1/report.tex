\documentclass[12pt,letterpaper]{article}

\usepackage{fullpage}
\usepackage{hyperref}
\usepackage{amsmath}
\usepackage{graphicx}


\begin{document}

\begin{flushright}
Kyle Cesare \\
Kevin Hess \\
Bryce Holley
\end{flushright}

{\center\textbf{Programming Assignment 1} \\}

\section{Pseudo-code}

\subsection{Brute-Force}

\begin{verbatim}
  for each element i in A:
      for each element j (starting with element i+1) in A:
          if A[i] > A[j]:
              inversions++
\end{verbatim}

\subsection{Naive Divide and Conquer}

\begin{verbatim}
  count_inversions(array A):
      if A.size is not even, insert 0 at beginning of A
      
      if A.size == 2:
          if A[1] > A[0]:
              return 1
          else:
              return 0
      
      array L = left half of A
      array R = right half of A
      
      for each element i in array L:
          for each element j in array R:
              if L[i] > R[j]:
                  inversion++
      
      return inversions + count_inversions(L) + count_inversions(R)
\end{verbatim}

\subsection{Merge and Count}
    
\begin{verbatim}
  count_inversions(array A):
      if A.size is not even, insert 0 at beginning of A
      
      if A.size == 2:
          if A[1] > A[0]:
              return 1
          else:
              return 0
      
      array LS = left half of A
      array RS = right half of A
      LS.sort
      RS.sort
      
      for each element i in RS:
          for each element j in LS:
              if LS[i] > RS[j]:
                  inversions += LS.size - i
                  break
      
      return inversions + count_inversions(L) + count_inversions(R)
\end{verbatim}

\section{Correctness Proof}
\subsection{Naive Divide and Conquer}
\begin{description}
  \item[Base Case] $n=2$ and $S = \{a, b\}$. If $a < b$, return $0$. If $b < a$,
    return $1$.
  \item[Inductive Hypothesis] Assume that for all inputs of size $2, 3,
    ... , n-1$ the algorithm returns the correct number of inversions.
  \item[Inductive Step] The algorithm iteratively compares every element of the
    right array with every element of the left array. Assume, for contradiction,
    that there is an inversion not counted by the algorithm such that $l_i >
    r_j$ where $l_i \in L$ and $r_j \in R$. This is a contradiction to the
    definition of our algorithm, because by the iteration of the two sets, this
    inversion would have been counted. The number of inversions is then the
    inversions found by iteration plus the inversions in the left array and the
    right array, by the Inductive Hypothesis.
\end{description}

\subsection{Merge and Count}
\begin{description}
  \item[Base Case] $n=2$ and $S = \{a, b\}$. If $a < b$, return $0$. If $b < a$,
    return $1$.
  \item[Inductive Hypothesis] Assume that for all inputs of size $2, 3,
    ... , n-1$ the algorithm returns the correct number of inversions.
  \item[Inductive Step] The left and right arrays are first sorted. For each
    element in the right array (index $j$), the algorithm iterates over every
    element in the left array (index $i$). It loops until $l_i > r_j$. At
    this point, because the arrays are sorted, there are $n = length(l) - i$
    elements remaining in the left array, of which every one is greater than
    $r_j$. Therefore, there are $n$ inversions for $r_j$.

    Assume, for contradiction, that there is an inversion $\{l_i, r_j\}$ that
    was not counted. By the definition of our algorithm, some element $l_{i-k} <
    r_j$ would have been found. Then, the remaining $n = length(l) - (i - k)$
    elements would have been counted as inversions. This is a contradiction to
    the assumption that $l_i$ was not marked.
\end{description}

\section{Analysis of Runtime}
\subsection{Brute Force}
\begin{align*}
  T(n) &= \sum_{i=0}^{n} i * c \\
       &= c \sum_{i=0}^{n} i \\
       &= \frac{cn(n+1)}{2} \\
\end{align*}

\subsection{Naive Divide and Conquer}
\begin{align*}
  T(n) &= c(\frac{n}{2})(\frac{n}{2}) + T(\frac{n}{2}) + T(\frac{n}{2}) \\
       &= c(\frac{n}{2})(\frac{n}{2}) + 2T(\frac{n}{2}) \\
       &= c(\frac{n}{2})(\frac{n}{2}) + 2[c(\frac{n}{4})(\frac{n}{4}) + 2T(\frac{n}{4})] \\
       &= c(\frac{n}{2})(\frac{n}{2}) + 2[c(\frac{n}{4})(\frac{n}{4}) + 2T[c(\frac{n}{8})(\frac{n}{8}) + 2T(\frac{n}{8})]] \\
       &= \sum_{i=1}^{\log_2 n} 2^{i-1} c (\frac{n}{2^i})^2 \\
       &= cn^2 \sum_{i=1}^{\log_2 n} 2^{-i-1} \\
       &= cn^2 \\
\end{align*}

\subsection{Merge and Count}
$S$ is the sort function, and $C$ is the comparison function.

\begin{align*}
  T(n) &= 2S(\frac{n}{2}) + C(\frac{n}{2}) + 2T(\frac{n}{2}) \\
       &= 2S(\frac{n}{2}) + C(\frac{n}{2}) + 2[2S\frac{n}{4} + C\frac{n}{4} + 2T\frac{n}{4}] \\
       &= \sum_{i=1}^{\log n} 2^i S(\frac{n}{2^i}) + 2^{i-1} C(\frac{n}{2^i}) \\
  S(n) &= n \log n , C(n) = 2n \\
  T(n) &= n(\log^2 n + \frac{\log n \log (n + 1)}{2} + \log n) \\
\end{align*}

\section{Testing}
Our algorithm found the following number of inversions for each line in the test
file: 252180, 250488, 243785, 247021, 250925, 256485, 249876, 253356, 255204,
247071.

\section{Extrapolation and Interpolation}

\begin{align*}
  \frac{n_1 \log n_1}{t_1} &= \frac{n_2 \log n_2}{t_2} \\
  n_2 \log n_2 &= \frac{t_2 n_1 \log n_1}{t_1}
\end{align*}

If $t_2 = 1 \text{ hour} = 3600 \text{ seconds}$, $t_1 = 10 \text{ seconds}$,
and $n_1$ = $10^5$, then $n_2 = 4.1 \times 10^8$.

Our log-log plot is roughly linear, which agrees with our asymptotic runtime.
We measured the slope to be roughly $m = 1.6$. Any discrepancy between this
result and our asymptotic runtime analysis are probably due to the program being
run on top of an operating system, in which it could be context switched at any
moment.

\includegraphics[width=\linewidth]{plot}

\end{document}
