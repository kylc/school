\documentclass[12pt,letterpaper]{article}

\usepackage{fullpage}
\usepackage{hyperref}
\usepackage{amsmath}
\usepackage{graphicx}


\begin{document}

\begin{flushright}
Kyle Cesare \\
Kevin Hess \\
Bryce Holley
\end{flushright}

{\center\textbf{Programming Assignment 2} \\}

\section{Pseudo-code}

\subsection{Brute-Force}

\begin{verbatim}
brute_force(A):
  max_sum = 0
  for i = 0..n
    this_sum = 0
    for j = i..n
      this_sum += A[j]
      best_sum = max(best_sum, this_sum)

  return best_sum
\end{verbatim}

\subsection{Divide and Conquer}

\begin{verbatim}
divide_conquer(A):
  midpoint = floor(n / 2)

  if n == 0
    return 0
  if n == 1
    return A[0]

  max_left = divide_conquer(A[0..midpoint])
  max_right = divide_conquer(A[(midpoint+1)..n])

  cur_sum = 0
  left_sum = 0
  for i = midpoint..0
    cur_sum += A[i]
    if cur_sum > left_sum
      left_sum = cur_sum

  cur_sum = 0
  right_sum = 0
  for i = (midpoint+1)..n
    cur_sum += A[i]
    if cur_sum > right_sum
      right_sum = cur_sum

  return max(max_left, max_right, left_sum + right_sum)
\end{verbatim}

\subsection{Dynamic Programming}
    
\begin{verbatim}
dynamic_programming(A):
  M[0] = max(0, A[0])
  best_sum = M[0]
  for i = 1..n
    if M[i - 1] + A[i] > 0
      M[i] = M[i - 1] + A[i]
      best_sum = max(M[i - 1] + A[i], best_sum)
    else
      M[i] = 0

  return best_sum
\end{verbatim}

\section{Correctness Proof}
\subsection{Divide and Conquer}
Proof by Induction:
\begin{description}
  \item[Base Case] If $n=0$, the algorithm returns $0$. If $n=1$, the algorithm
    returns $\max \{ 0, A(1) \}$.
  \item[Inductive Hypothesis] Assume that our algorithm finds the maximum
    subarray for any array $A$ of any size $\{1,\ldots,n-1\}$.
  \item[Inductive Step] For any array $A$ of size $n$, we show that the
    algorithm returns the correct value. There are three possible locations of
    the maximum subarray in $A$.

    \begin{description}
      \item[Exclusively in Left Half] By the inductive hypothesis, the algorithm
        correctly finds the maximum subarray.
      \item[Exclusively in Right Half] By the inductive hypothesis, the algorithm
        correctly finds the maximum subarray.
      \item[Includes Center Element] The maximum subarray must include the
        midpoint element. Therefore, the value of the maximum subarray must be
        the sum of the maximum subarray radiating leftward from the center and
        the maximum subarray radiating rightward from the center, plus the
        center element itself. These values are found iteratively.

        That is, given the midpoint index $i$, the maximum subarray sum is the
        sum of the maximum subarray iterating leftward from $i-1$ and the
        sum of the maximum subarray iterating rightward from $i$.
    \end{description}
\end{description}

\subsection{Dynamic Programming}
Proof by Induction:
\begin{description}
  \item[Base Case] If $n=0$, the sum of an empty array is zero, and max is
    initially defined as $0$. If $n=1$, there are two cases:
    \begin{description}
      \item[$A(1) \leq 0$:] The initial max of $0$ is returned. This is correct
        as the array contains only a single negative number.
      \item[$A(1) > 0$:] The max is set to n, correctly returning the only
        element in the array as the largest possible subarray.
    \end{description}
  \item[Inductive Hypothesis] Assume that for all $i=1,\ldots,n-1$ the function
    returns the sum of the maximum subarray.
  \item[Inductive Step]

    Adding element $n$ to the array means that the for loop will have to go
    through one additional iteration.  By the inductive hypothesis, the max has
    already correctly been set for the $n - 1$ element. Thus, the only way the
    additional element could change the max is if the max subarray ends at the
    $n$th element.
        
    Let $M(i)$ denote the sum of the maximum subarray of $A$ ending at index
    $i$.  There are two cases for the $n$th element:
    
    \begin{description}
      \item[$A(n) < -M(n-1)$:] In this case, the addition of element $n$ to
        the sequence would make the sum negative. Therefore, this cannot be
        used, as a more optimal subarray would be the empty array. So,
        we set $M(n) = 0$.

      \item[$A(n) \geq -M(n-1)$:] In this case, the addition of element $n$ will
        result in a positive sum. Therefore, this element is possibly included
        in the maximum subarray. We may then set $M(n) = M(n-1) + A(n)$.
    \end{description}

    Now that we have values for $M(i)$ for $i=1,\ldots,n$, the overall maximum
    subarray sum of the array $A$ is $\max \{ M(1), \ldots, M(n) \}$.
\end{description}

\section{Analysis of Runtime}
\subsection{Brute Force}
\begin{align*}
  T(n) &= cn + c(n-1) + c(n-2) + ... + 2c + c \\
       &= \sum_{i=0}^{n} ci \\
       &= c \sum_{i=0}^{n} i \\
       &= \frac{cn(n+1)}{2} \\
       &= O(n^2) \\
\end{align*}

\subsection{Divide and Conquer}
\begin{align*}
  T(n) &= T(\frac{n}{2}) + T(\frac{n}{2}) + \frac{cn}{2} + \frac{cn}{2} + d \\
       &= 2T(\frac{n}{2}) + cn + d \\
       &= 4T(\frac{n}{4}) + 2cn + 3d \\
       &= 8T(\frac{n}{8}) + 3cn + 7d \\
       &= 2^k T(\frac{n}{2^k}) + kcn + (2^k - 1)d \\
\end{align*}

Because we continually divide our input in half, our recursion will end at $k =
\log_2 n$.

\begin{align*}
  T(n) &= 2^{\log_2 n}T(\frac{n}{2^{\log_2 n}}) + cn\log_2 n + (2^{\log_2 n} - 1)d \\
       &= nT(1) + cn\log_2 n + (n - 1)d \\
       &= cn\log_2 n + cn + d \\
       &= O(n\log n) \\
\end{align*}

\subsection{Dynamic Programming}

\begin{align*}
  T(n) &= cn \\
       &= O(n) \\
\end{align*}

\section{Testing}
Our algorithm found the following results for each line in the test file: 4711,
8932, 11242, 22557, 9097, 23875, 6145, 6957, 13734, 14808, 7654, 7723, 5877,
8898, 10265, 10285, 21605, 5396, 16430, 8929, 

\section{Empirical Analysis}

As is shown in the graph below, the brute force algorithm quickly becomes
unusable for large input sizes. This is due to its $O(n^2)$ complexity. For very
small input sizes, it may be usable because of its ease of implementation.

The divide and conquer algorithm is also fairly easy to reason about. The
resulting complexity is $O(n\log n)$, which is acceptable for large input sizes.
However, its use of recursion is a potential pitfall. Our naive implementation
is not tail recursive, so for input sizes too large it would overflow the stack.

The dynamic programming solution is by far the most efficient. It is iterative
rather than recursive, so there is no need to worry about overflowing the stack.
The complexity is $O(n)$, which is much better than the brute force algorithm
and still better than the divide and conquer algorithm. However, the
implementation is fairly nuanced and non-obvious.

\includegraphics[width=\linewidth]{plot}

\end{document}
