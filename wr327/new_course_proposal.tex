\documentclass[12pt,letterpaper]{article}

\usepackage{ifpdf}
\usepackage[utf8]{inputenc}
\usepackage{times}
\usepackage{graphicx}
\usepackage{setspace}
\usepackage{float}

\usepackage{anysize}
\marginsize{1in}{1in}{1in}{1in} 
% Resize section titles
\usepackage{titlesec}
\titleformat{\section}{\large\bfseries}{\thesection}{1em}{}
\titleformat{\subsection}{\bfseries}{\thesection}{1em}{}

% "<author> <page>" style page headers
\usepackage{fancyhdr}
\fancypagestyle{norule}{ %
    \renewcommand{\headrulewidth}{0pt}
    \renewcommand{\footrulewidth}{0pt}
}
\fancyhf{}
\pagestyle{norule}
\rhead{Cesare \thepage}
\setlength\headsep{0.333in}

% Works cited environment
% (to start, use \begin{workscited...}, each entry preceded by \bibent)
%  - from Ryan Alcock's MLA style file
\newcommand{\bibent}{\noindent \hangindent 40pt}
\newenvironment{workscited}{\newpage\begin{center} \large\bfseries Works Cited \end{center} \doublespacing}{\newpage}

\DeclareGraphicsExtensions{.png}

\begin{document}

% Header
\begin{flushleft}
  Kyle Cesare \\
  WR327, Section 016\\
  March 9, 2013
\end{flushleft}

% No header on the first page
\thispagestyle{empty}

\begin{center}
  \Large \textbf{CS451: Building Open Source Software}
\end{center}

\section*{Introduction}

\section*{Background}

\section*{Course Information}

\begin{tabular}{ | l | l | }
  \hline
  Department           & Electrical Engineering and Computer Science \\ \hline
  Course Number        & CS451 \\ \hline
  Disciplines Involved & Computer Science, Communication, Business \\ \hline
  Pre-requisites       & CS261, CS311 \\ \hline
\end{tabular}

% Department, Course number, description of course contents, how it fits into
% the curriculum, disciplines involved, pre-requisites.
% Reasons for offering this course. Degree fulfillments (bacc core, major core,
% WIC, project, etc.). Academic/Professional fulfillments: Who the course would
% appeal to and how it would meet their academic/professional needs.
\subsection*{Description}
\begin{description}
  \item[Test] A test outcome.
\end{description}

% Describe the pedagogical outcomes the course will meet.
\subsection*{Outcomes}

\subsection*{Schedule}
\begin{tabular}{| c | l | }
  \hline
  \textbf{Week}  & \textbf{Material} \\ \hline
  1              & This is a test week. \\ \hline
  2              & This is a test week. \\ \hline
  3              & This is a test week. \\ \hline
  4              & This is a test week. \\ \hline
  5              & This is a test week. \\ \hline
  6              & This is a test week. \\ \hline
  7              & This is a test week. \\ \hline
  8              & This is a test week. \\ \hline
  9              & This is a test week. \\ \hline
  10             & This is a test week. \\ \hline
\end{tabular}

\subsection*{Resources}
\begin{itemize}
  \item Some resources
  \item Some resources
  \item Some resources
\end{itemize}

\section*{Conclusion}

\begin{workscited}

\end{workscited}
\end{document}
