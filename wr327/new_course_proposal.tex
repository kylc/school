\documentclass[12pt,letterpaper]{article}

\usepackage{ifpdf}
\usepackage[utf8]{inputenc}
\usepackage{times}
\usepackage{graphicx}
\usepackage{setspace}
\usepackage{tabulary}
\usepackage{float}
\usepackage{hyperref}
\usepackage{array}

\usepackage{anysize}
\marginsize{1in}{1in}{1in}{1in} 

% Works cited environment
% (to start, use \begin{workscited...}, each entry preceded by \bibent)
%  - from Ryan Alcock's MLA style file
\newcommand{\bibent}{\noindent \hangindent 40pt}
\newenvironment{workscited}{\newpage\begin{center} \Large\bfseries Works Cited \end{center} \doublespacing}{\newpage}

\DeclareGraphicsExtensions{.png}

\begin{document}

\begin{titlepage}
  \begin{center}
    {\LARGE Proposal of New E-Campus Course:\\CS451 Building Open Source Software} \\[2cm]

    \begin{minipage}{0.4\textwidth}
      \begin{flushleft}
        \emph{Submitter:} \\
        Kyle Cesare \\
        Oregon State University \\
      \end{flushleft}
    \end{minipage}
    \begin{minipage}{0.4\textwidth}
      \begin{flushleft}
        \emph{Recipient:} \\
        Executive Director Lisa L. Templeton \\
        Oregon State University \\
      \end{flushleft}
    \end{minipage}

    \vfill

    March 15, 2013

  \end{center}
\end{titlepage}

% Page number on the right, starting with 2
\pagestyle{myheadings}
\pagenumbering{roman}
\setcounter{page}{2}

% Don't number the sections
\setcounter{secnumdepth}{0}

\section{Abstract}
In CS451 Building Open Source Software, students will learn the tools and
processes used in open source software.  Students are currently limited to
working only with small purpose-made teaching software, but teaching students
about open source would open up thousands of real-world software projects.  Open
source is also a great way for students to learn about cutting-edge technology.
Open source also gives students' careers a huge boost, as many companies look
for open source participation and contribution when hiring.  In this course,
students will first be introduced to open source.  They will then learn about
version control systems, communication tools, and development patterns.  Then,
they will engage in a project studying two large open source projects.  For
these reasons, CS451 should be added to the E-Campus curriculum.

\pagebreak

\tableofcontents

\cleardoublepage
\addcontentsline{toc}{section}{\listfigurename}
\listoftables

\pagebreak

\pagenumbering{arabic} 
\section{Introduction}
In CS451 Building Open Source Software, students will learn the tools and
processes commonly used in open source and free software projects.  I propose
this course be added to the E-Campus curriculum to teach students the skills
required to contribute to real-world projects working with other developers.
CS451 will not attempt to to cover every development tool and process.  Rather,
it will discuss a selection of the most popular tools and processes.

This proposal is organized as follows: first, I will discuss the background of
related coursework already being offered.  Then, I will provide information on
the proposed course, including a high-level description, a set of learning
outcomes, a 10-week schedule of topics, and a list of possible resources for the
class.

\section{Background}
% TODO: Citation
Open source software is software to which the source code is freely available.
This is opposed to proprietary software.  Some examples of open source software
include the Firefox web browser, OpenOffice, and the Linux operating system
(TODO: FSF).

While there are many computer science courses available on E-Campus, very few
allow students to analyze and understand existing projects.  While CS361 and
CS362 Software Engineering teach development methodologies and tools, they do so
on artificial software contrived specifically for the class.  Studying open
source projects would expose students to the development of large real-world
software.

% TODO: Expand.  Talk about pet projects and resumes.
% TODO: Citations needed
Open source is also a great way for students to learn about cutting-edge
technologies and ideas.  For example, the open source programming language
Haskell is considered one of the most technologically advanced programming
languages, implementing a number of very recent research topics.  Because it is
open source, researchers, students, and volunteers can all use it as a learning
tool and even make contributions.

% TODO: Citation
There are also many career opportunities stemming from open source development.
For example, Mozilla, the company behind Firefox, clearly states on their hiring
page that they are seeking software engineers with experience in participating
and contributing to open source projects (Mozilla).  Google also states that
open source experience is beneficial, and they even hire developers to
exclusively work on open source projects (Google).

\section{Course Information}

\begin{table}[H]
  \begin{center}
    \renewcommand{\arraystretch}{1.5}
    \begin{tabular}{ | l | l | }
      \hline
      \textbf{Department}
        & Electrical Engineering and Computer Science \\ \hline
      \textbf{Course Number}
        & CS451 \\ \hline
      \textbf{Course Name}
        & Building Open Source Software\\ \hline
      \textbf{Disciplines Involved}
        & Computer Science, Communication, Business \\ \hline
      \textbf{Pre-requisites}
        & CS261, CS311, CS361, CS362 \\ \hline
    \end{tabular}
    \caption{Course Information}
  \end{center}
\end{table}

% Department, Course number, description of course contents, how it fits into
% the curriculum, disciplines involved, pre-requisites.
% Reasons for offering this course. Degree fulfillments (bacc core, major core,
% WIC, project, etc.). Academic/Professional fulfillments: Who the course would
% appeal to and how it would meet their academic/professional needs.
\subsection{Description}
In CS451 Building Open Source Software, students will learn the tools and
processes used in open source software.  Students currently learn many of the
techniques used in building proprietary software, but open source software often
uses entirely different practices.  This class is not strictly limited to
computer science majors.  Electrical and computer engineering majors would also
benefit from learning how to contribute to open source projects.

% Describe the pedagogical outcomes the course will meet.
\subsection{Outcomes}
Upon completion of the course, students should be proficient in the following
areas:

\begin{description}
  \item[Version Control] Use of version control systems, centralized and
    distributed, for the process of making change sets and committing them to
    project maintainers.
  \item[Communication Tools] Use of communication tools common in open source
    projects, allowing students to communicate with project maintainers and
    other developers.
  \item[Open Source Process] Understand the open source process, from project
    creation to management of large projects with many developers.
  \item[Contributing to Open Source] Understand the process of contributing to
    open source projects as a developers.
\end{description}

\subsection{Schedule}
\begin{table}[H]
  \begin{center}
    \renewcommand{\arraystretch}{1.5}
    \begin{tabulary}{0.8\textwidth}{| c | L | }
      \hline
      \textbf{Week}  & \textbf{Material} \\ \hline
      1              & Introduction to Open Source \\ \hline
      2              & Communication Tools \\ \hline
      3              & Version Control Systems (CVS, Subversion) \\ \hline
      4              & Version Control Systems (Git, Mercurial) \\ \hline
      5              & Open Source Development Patterns \\ \hline
      6              & Open Source Development Patterns Cont. \\ \hline
      7              & Project Analysis: Mozilla Firefox \\ \hline
      8              & Project Analysis: Mozilla Firefox Cont. \\ \hline
      9              & Project Analysis: Linux Kernel \\ \hline
      10             & Project Analysis: Linux Kernel Cont. \\ \hline
    \end{tabulary}
    \caption{10-Week Schedule of Topics}
  \end{center}
\end{table}

\subsection{Resources}
\begin{description}
  \item[Open Source Wikibook (\url{http://en.wikibooks.org/wiki/Open_Source})]
    This Wikibook provides an overview of the history of open source, open
    source philosophy, and the common licensing methods.
  \item[Free Software Foundation (\url{http://fsf.org})] The Free Software
    Foundation provides a number of resources pertaining to open source
    licensing and free software services.  Students will be able to learn about
    free software licensing with the Free Software Foundation's plain English
    explanations.
  \item[GNU Operating System (\url{http://www.gnu.org/})] One of the largest
    open source projects: an entire operating system implementation.
  \item[GitHub (\url{http://github.com})] A directory of open-source projects.
    Students can explore open source, from small personal experiments to giant
    open source projects with hundreds of developers.  Students can also use the
    integrated issue tracker and associated tools to learn about communication
    within open source projects.
\end{description}

\section{Conclusion}
To summarize, CS451 Building Open Source Software is a class focused on teaching
students about the tools and processes used in open source and free software
projects.  Students would be able to work with large real-world software,
something they wouldn't otherwise be able to do.  Students will learn about the
latest technologies as they pertain to software development.  Students will have
useful open source experience that modern employers are seeking.  For the
reasons outlined above, CS451 Building Open Source Software should be added to
the E-Campus curriculum.

\begin{workscited}
\addcontentsline{toc}{section}{Works Cited}

% http://www.fsf.org/about/what-is-free-software

% http://careers.mozilla.org/en-US/position/ozWfXfwr

% https://www.google.com/about/jobs/search/#!t=jo&jid=29119&

\end{workscited}
\end{document}
