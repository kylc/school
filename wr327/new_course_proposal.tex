\documentclass[12pt,letterpaper]{article}

\usepackage{ifpdf}
\usepackage[utf8]{inputenc}
\usepackage{times}
\usepackage{graphicx}
\usepackage{setspace}
\usepackage{tabulary}

\usepackage{anysize}
\marginsize{1in}{1in}{1in}{1in} 

% Works cited environment
% (to start, use \begin{workscited...}, each entry preceded by \bibent)
%  - from Ryan Alcock's MLA style file
\newcommand{\bibent}{\noindent \hangindent 40pt}
\newenvironment{workscited}{\newpage\begin{center} \Large\bfseries Works Cited \end{center} \doublespacing}{\newpage}

\DeclareGraphicsExtensions{.png}

\begin{document}

\begin{titlepage}
  \begin{center}
    {\LARGE Proposal of New eCampus Course:\\CS451 Building Open Source Software} \\[2cm]

    \begin{minipage}{0.4\textwidth}
      \begin{flushleft}
        \emph{Submitter:} \\
        Kyle Cesare \\
        Oregon State University \\
      \end{flushleft}
    \end{minipage}
    \begin{minipage}{0.4\textwidth}
      \begin{flushleft}
        \emph{Recipient:} \\
        Executive Director Lisa L. Templeton \\
        Oregon State University \\
      \end{flushleft}
    \end{minipage}

    \vfill

    March 15, 2013

  \end{center}
\end{titlepage}

% Page number on the right, starting with 2
\pagestyle{myheadings}
\pagenumbering{roman}
\setcounter{page}{2}

% Don't number the sections
\setcounter{secnumdepth}{0}

\section{Abstract}

\pagebreak

\tableofcontents

\cleardoublepage
\addcontentsline{toc}{section}{\listfigurename}
\listoftables

\pagebreak

\pagenumbering{arabic}

\section{Introduction}

\section{Background}

\section{Course Information}

\begin{table}[h]
  \begin{center}
    \begin{tabular}{ | l | l | }
      \hline
      \textbf{Department}
        & Electrical Engineering and Computer Science \\ \hline
      \textbf{Course Number}
        & CS451 \\ \hline
      \textbf{Disciplines Involved}
        & Computer Science, Communication, Business \\ \hline
      \textbf{Pre-requisites}
        & CS261, CS311, CS361, CS362 \\ \hline
    \end{tabular}
    \caption{Course Information}
  \end{center}
\end{table}

% Department, Course number, description of course contents, how it fits into
% the curriculum, disciplines involved, pre-requisites.
% Reasons for offering this course. Degree fulfillments (bacc core, major core,
% WIC, project, etc.). Academic/Professional fulfillments: Who the course would
% appeal to and how it would meet their academic/professional needs.
\subsection{Description}

% Describe the pedagogical outcomes the course will meet.
\subsection{Outcomes}
\begin{description}
  \item[Test] A test outcome.
\end{description}

\subsection{Schedule}

\begin{table}[h]
  \begin{center}
    \begin{tabulary}{0.8\textwidth}{| c | L | }
      \hline
      \textbf{Week}  & \textbf{Material} \\ \hline
      1              & This is a test week. \\ \hline
      2              & This is a test week. \\ \hline
      3              & This is a test week. \\ \hline
      4              & This is a test week. \\ \hline
      5              & This is a test week. \\ \hline
      6              & This is a test week. \\ \hline
      7              & This is a test week. \\ \hline
      8              & This is a test week. \\ \hline
      9              & This is a test week. \\ \hline
      10             & This is a test week. \\ \hline
    \end{tabulary}
    \caption{10-Week Schedule}
  \end{center}
\end{table}

\subsection{Resources}
\begin{itemize}
  \item Some resources
  \item Some resources
  \item Some resources
\end{itemize}

\section{Conclusion}

\begin{workscited}
\addcontentsline{toc}{section}{Works Cited}

\end{workscited}
\end{document}
