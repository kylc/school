\documentclass[12pt,letterpaper]{article}

\usepackage{ifpdf}
\usepackage[utf8]{inputenc}
\usepackage{times}
\usepackage{graphicx}
\usepackage{setspace}
\usepackage{tabulary}
\usepackage{float}
\usepackage{hyperref}

\usepackage{anysize}
\marginsize{1in}{1in}{1in}{1in} 

% Works cited environment
% (to start, use \begin{workscited...}, each entry preceded by \bibent)
%  - from Ryan Alcock's MLA style file
\newcommand{\bibent}{\noindent \hangindent 40pt}
\newenvironment{workscited}{\newpage\begin{center} \Large\bfseries Works Cited \end{center} \doublespacing}{\newpage}

\DeclareGraphicsExtensions{.png}

\begin{document}

\begin{titlepage}
  \begin{center}
    {\LARGE Proposal of New eCampus Course:\\CS451 Building Open Source Software} \\[2cm]

    \begin{minipage}{0.4\textwidth}
      \begin{flushleft}
        \emph{Submitter:} \\
        Kyle Cesare \\
        Oregon State University \\
      \end{flushleft}
    \end{minipage}
    \begin{minipage}{0.4\textwidth}
      \begin{flushleft}
        \emph{Recipient:} \\
        Executive Director Lisa L. Templeton \\
        Oregon State University \\
      \end{flushleft}
    \end{minipage}

    \vfill

    March 15, 2013

  \end{center}
\end{titlepage}

% Page number on the right, starting with 2
\pagestyle{myheadings}
\pagenumbering{roman}
\setcounter{page}{2}

% Don't number the sections
\setcounter{secnumdepth}{0}

\section{Abstract}

\pagebreak

\tableofcontents

\cleardoublepage
\addcontentsline{toc}{section}{\listfigurename}
\listoftables

\pagebreak

\pagenumbering{arabic}

\section{Introduction}
In CS451, students will learn the tools and processes used in open source
software.  I propose this course be added to the eCampus cirriculum to teach
students the skills required to contribute to real-world projects working with
other developers.  CS451 will not attempt to to cover every development tool and
process.  Rather, it will discuss a selection of the most popular tools and
processes.

This proposal is organized as follows: first, I will discuss the background of
related coursework already being offered.  Then, I will provide information on
the proposed course, including a high-level description, a set of learning
outcomes, a 10-week schedule of topics, and a list of possible resources for the
class.

\section{Background}

\section{Course Information}

\begin{table}[h]
  \begin{center}
    \begin{tabular}{ | l | l | }
      \hline
      \textbf{Department}
        & Electrical Engineering and Computer Science \\ \hline
      \textbf{Course Number}
        & CS451 \\ \hline
      \textbf{Course Name}
        & Building Open Source Software\\ \hline
      \textbf{Disciplines Involved}
        & Computer Science, Communication, Business \\ \hline
      \textbf{Pre-requisites}
        & CS261, CS311, CS361, CS362 \\ \hline
    \end{tabular}
    \caption{Course Information}
  \end{center}
\end{table}

% Department, Course number, description of course contents, how it fits into
% the curriculum, disciplines involved, pre-requisites.
% Reasons for offering this course. Degree fulfillments (bacc core, major core,
% WIC, project, etc.). Academic/Professional fulfillments: Who the course would
% appeal to and how it would meet their academic/professional needs.
\subsection{Description}
In Building Open Source Software (CS451), students will learn the tools and
processes used in open source software.  Students currently learn many of the
techniques used in building proprietary software, but open source software often
uses entirely different practices.  This class is not strictly limited to
computer science majors.  Electrical and computer engineering majors would also
benefit from learning how to contribute to open source projects.

% Describe the pedagogical outcomes the course will meet.
\subsection{Outcomes}
Upon completion of the course, students should be proficient in the following
areas:

\begin{description}
  \item[Version Control] Use of version control systems, centralized and
    distributed, for the process of making change sets and committing them to
    project maintainers.
  \item[Communication Tools] Use of communication tools common in open source
    projects, allowing students to communicate with project maintainers.
  \item[Open Source Process] Understand the open source process, from project
    creation to management of large projects with many developers.
\end{description}

\subsection{Schedule}
\begin{table}[H]
  \begin{center}
    \begin{tabulary}{0.8\textwidth}{| c | L | }
      \hline
      \textbf{Week}  & \textbf{Material} \\ \hline
      1              & Introduction to Open Source \\ \hline
      2              & Communication Tools \\ \hline
      3              & Version Control Systems (CVS, Subversion) \\ \hline
      4              & Version Control Systems (Git, Mercurial) \\ \hline
      5              & Open Source Development Patterns \\ \hline
      6              & Open Source Development Patterns Cont. \\ \hline
      7              & Project Analysis: Mozilla Firefox \\ \hline
      8              & Project Analysis: Mozilla Firefox Cont. \\ \hline
      9              & Project Analysis: Linux Kernel \\ \hline
      10             & Project Analysis: Linux Kernel Cont. \\ \hline
    \end{tabulary}
    \caption{10-Week Schedule}
  \end{center}
\end{table}

\subsection{Resources}
\begin{description}
  \item[Free Software Foundation (\url{http://fsf.org})] The Free Software
    Foundation provides a number of resources pertaining to open source
    licensing and free software services.
  \item[Open Source Wikibook (\url{http://en.wikibooks.org/wiki/Open_Source})]
    This Wikibook provides an overview of the history of open source, open
    source philosophy, and the common licensing methods.
  \item[GitHub (\url{http://github.com})] A directory of open-source projects.
\end{description}

\section{Conclusion}

\begin{workscited}
\addcontentsline{toc}{section}{Works Cited}

\end{workscited}
\end{document}
