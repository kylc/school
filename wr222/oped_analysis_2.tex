\documentclass[12pt,letterpaper]{article}

\usepackage{ifpdf}
\usepackage{mla}
\usepackage[utf8]{inputenc}

\begin{document}
\begin{mla}{Kyle}{Cesare}{Roush}{WR222}{October 27, 2011}{Are Corporations
People?}

% -- SECTION 1 --
% Summary
In Gary Gutting's New York Times article titled ``Corporations, People and
Truth,'' published on October 12, 2011, the author argues that corporations are
not people.  This is in response to a recent quote from presidential candidate
Mitt Romney stating that ``Corporations are people'' (Gutting).  According to
Gutting, corporations are, in fact, not people.  He agrees that, pedantically,
corporations are ``the people who own, run, and work for them.'' (Gutting)
However, he counters that a corporation cannot feel human emotions, cannot
``fall in love, write poetry or be depressed.'' (Gutting)  He points out that
corporations hold no moral standing in our society because ``corporations have
no core dedication to fundamental human values.'' (Gutting)  He goes even
further, making the claim that corporations are a threat to truth because of
their motivation only for profit.
% TODO: Finish this, use last two paragraphs of article

% -- SECTION 2 --
% Clearly explain how this is an argument of definition
It is clear that Gutting is trying to define the words ``corporation'', and test
whether or not that definition is compatible with the definition of ``person.''
Gutting provides two main definitions: that provided by Romney, and his own.
Romney's definition is, according to the author, a pedantic take on the word.
The author offhandedly admits that, sure, a corporation is simply the sum of the
people working under it, but only when looked at in the simplest of contexts.
However, corporations, unlike the people who work for them, are driven solely by
profits.  They are in no way motivated by emotion or compassion for other human
beings.  If it is advantageous for a company to admit to a mistake, it will do
so not because it's the right thing to do, but because there is future profit
opportunity by doing so.

I think that Gutting makes a very compelling argument for a very important
question in U.S. politics today.  Romney's definition seems far too simplistic,
not taking into account any external factors.  As Gutting points out, it is the
literal definition of a corporation, but a corporation as a whole acts not based
on the sum of the ideals held within the corporation, but solely on motivation
for profit.
% TODO: What do you think about the meaning or definition being argued?

% -- SECTION 3 --
% Defining strategies
% - Compare and contrast his definition to Romney's
% - Definition by example - most apparent in advertising
% - Definition by example - J&J
Gutting uses a few strategies to relay his definition to the reader.  The most
obvious is his comparing and contrasting to Romney's definition.  Given that
Romney's is such a simple definition, the author is mostly contrasting the two
definitions.  Gutting makes his position clear: he thinks that corporations are
not people.  He also makes use of a very compelling example.  He mentions a
story in which ``seven people in Chicago died from poisoned Tylenol, Johnson \&
Johnson appealed to its credo, which makes concern for its customers a primary
corporate goal, and told the entire truth about what had happened.'' (Gutting)
The reader might initially think that this goes against Gutting's argument, but
he catches the reader when he states the profit motives of such a move.  The
article relies most heavily on operational definitions.  Gutting states all the
bad things corporations do that people don't: people ``support decision we see
as right even if they work to our disadvantage'' (Gutting) but corporations
don't.  Corporations ``have no core dedication to fundamental human values,''
(Gutting) while humans, obviously, do.  The combination of these strategies
makes the argument that corporations cannot be defined as people very
convincing.


\begin{workscited}

\bibent Gutting, Gary. "Corporations, People and Truth." \emph{The New York
Times}. NYTimes, 12 Oct. 2011. Web. 25 Oct. 2011. 

\end{workscited}
\end{mla}
\end{document}
