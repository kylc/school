\documentclass[12pt,letterpaper]{article}

\usepackage{ifpdf}
\usepackage{mla}
\usepackage[utf8]{inputenc}

\begin{document}
\begin{mla}{Kyle}{Cesare}{Roush}{WR222}{October 7, 2011}{Precision Tracking
Without a Warrant}

% -- SECTION 1 --
% Summary (about half a page)
% - Author, title, date, publication name
% - United States v. Jones - GPS placed on suspected drug dealer
%   - No warrant
%   - In violation of Fourth Amendment?
% - Once expected anonymity in public, no longer an option?
% - Judge Ginsburg: not like more primitive technologies, different from
%   previous rulings the year before
% - Online tracking: facial recognition at Facebook
% - Google: within a few years, live feeds all over world, totally archived
% - Proposed "Geolocation Privacy and Surveillance Act"
%   - Requires warrant before acquiring geolocational info
%   - Requires consent before private companies give up geolocational info
% - Public life transformed forever if this fails
Jefferey Rosen, in his September 12, 2011 article titled ``Protect Our Right to
Anonymity,'' published in the \emph{New York Times}, argues that law enforcement
should not be able to use GPS or other tracking devices without first acquiring
a warrant.  He is chiefly concerned with the \emph{United States v. Jones}.
Police ``tracked [the victim's] movements for a month and used the
information to convict'' (Rosen 1) without first obtaining a warrant.  Is such
an action in violation of the Fourth Amendment?

Some say that ``we have no expectation of privacy when we are in public places''
(Rosen 1), and that these new technologies only make such tracking easier.
Justice Ginsburg counters that privacy is expected over a long period of time,
as ``the likelihood anyone will observe all those movements is effectively
nil.'' (1)  She expands, stating that this case is not like a group of similar
cases in the past few years, because those were mainly concerned with the
privacy over a single car trip, for example.  This case is concerned with a much
more prying eye.

% Technology is also playing a role, with Facebook rolling out facial recognition
% features that automatically tag people in photos, and Google predicting a global
% surveillance system that sounds awfully similar to Orwell's dystopian
% \emph{1984}.
% 
% Rosen closes his article stating that the \emph{Geolocation Privacy and
% Surveillance Act}, proposed by Senator Wyden and Representative Chaffetz to make
% unauthorized GPS use illegal for police, must pass.  If it fails, ``public life
% may be transformed in ways we can only begin to imagine.'' (Rosen 2).

% TODO: DISCUSS CRITERIA
The author's criteria for proper handling of this delicate issue is quite
simple.  Rosen wants a decision that will maximize personal liberties, and
conform to the Fourth Amendment as he knows it.  He doesn't want to add to the
tracking powers of police forces; rather he wants it to be illegal to track
people in such a granular manner.

% -- SECTION 2 --
% Analyze for audience (majority of time spent on this)
% - NYTimes, global reach
% - Audience:
%   - Those who aren't aware of latest tracking tech, cases
%   - Those concerned with personal freedoms
%   - Those looking for a solution
% - Warrants:
%   - Ubiquitous surveillance technology is bad
% - Pathos:
%   - ``our public life may be transformed in ways we can only begin to
%       imagine.'' (2)
%   - ``the expectations of anonymity in public that Americans have long taken
%      for granted'' (2)
%   - ``Americans will no longer be able to expect the same degree of anonymity
%       in public places that they have rightfully enjoyed since the founding
%       era'' (1)
% - Ethos:
%   - ``a law Professor at George Washington University'' (2)
% - Logos:
%    - ``Two federal appellate courts have upheld the use of GPS devices without
%      warrants in similar cases'' (1)
This article was published in the New York Times, a reputable news outlet with
global reach.  The audience is quite large.  First, it includes anyone who isn't
aware of the latest tracking technology and how it is being used.  Rosen
discusses ``technological enhancements like a GPS device'' (Rosen 1) that
provide much more granular results than, say, ``a beeper to help the police
follow a car for a 100-mile trip.'' (1)  Another clearly defined target audience
includes those who are concerned with personal freedoms.  Rosen provides plenty
of ammunition and emotional provocation with statements like ``our public life
may be transformed in ways we can only begin to imagine'' (2).  Finally, Rosen
targets those looking for a solution.  He makes readers aware of the
\emph{Geolocation Privacy and Surveillance Act}, ``which would provide federal
protection against public surveillance.'' (2)

Rosen also makes judicious use of the appeals.  He mainly uses pathos to access
the reader's emotional side, stating that we might lose ``the experience of
anonymity in public that Americans have long taken for granted'' (Rosen 2) and,
essentially, repeating himself with ``Americans will no longer be able to expect
the same degree of anonymity in public places that they have rightfully enjoyed
since the founding era'' (1).  In both of these cases, he's trying to get the
reader riled up in protection of his or her freedom.  He also brings out some
logos in these claims, citing that we have had such a right to anonymity for a
very long time, so why should it not continue?  He also uses logos when he
states ``a GPS device \dots is therefore qualitatively different than the more
limited technologically enhanced public surveillance that the Supreme Court has
upheld in the past'' (1).  Rosen uses very little ethos, only pointing out at
the very end that he is ``a law Professor at George Washington University'' (2).

% -- SECTION 3 --
% Reflection (one paragraph)
I believe that the argument made by Rosen was very effective.  I would qualify
myself as part of the target audience, and I felt that the strong use of
emotional appeal is very convincing when talking about such an
emotionally-linked topic as one's personal freedoms.  One weakness in his
argument is his credibility.  Besides his short informational snippet at the end
stating that he is a professor at George Washington University, he doesn't
introduce much information in the article as to why one should actually be
listening to him.

\begin{workscited}

\bibent Rosen, Jeffrey. ``Protecting Our Right to Anonymity.'' Editorial. The
New York Times 12 Sept. 2011: A31. Web. 3 Oct. 2011. 

\end{workscited}
\end{mla}
\end{document}
