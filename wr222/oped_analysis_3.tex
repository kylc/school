\documentclass[12pt,letterpaper]{article}

\usepackage{ifpdf}
\usepackage{mla}
\usepackage[utf8]{inputenc}

\begin{document}
\begin{mla}{Kyle}{Cesare}{Roush}{WR222}{November 21, 2011}{Stop the Great
Firwall of America}

% -- SECTION 1 --
% Summary
In Rebecca MacKinnon's New York Times article titled ``Stop the Great Firewall
of America'' published on November 15, 2011, the author argues that the Protect
IP Act would have disastrous consequences on startup businesses, Internet
service providers, and social networks such as Facebook and Twitter.  The act
would allow the Attorney General to create and modify a ``blacklist of sites to
be blocked by Internet service providers, search engines, payment providers and
advertising networks, all without a court hearing or a trial.'' (MacKinnon)  It
would also require websites be held for content created by their users.  For
instance, under the Digital Millenium Copyright Act, a website only needs to
remove material if the author of the content notifies them of the infringement.
With the Protect IP Act, the websites would have to be more proactive in
censoring content, as is done in China.  This creates ``huge overhead spending
by Internet companies for staff and technologies dedicated to monitoring users
and censoring any infringing material from being posted or transmitted.''
(MacKinnon)

% -- SECTION 2 --
% Clearly explain how this is a causal argument
The author states a cause, the Protect IP Act, and then traverses through the
list of possible effects.  This strategy becomes apparent when she states that
``the Protect IP Act... [threatens] to inflict collateral damage on democratic
discourse and dissent both at home and around the world.'' (MacKinnon)  We can
clearly see that she is going from the cause and mapping out a list of possible
effects.  These effects range from the trouble it will cause for Internet
providers as far as censoring content before it even hits the web to the
possible conflicts with free speech.

The author provides ample high-quality evidence.  Her use of China as an example
of the possible effects of establishing this black-listing precedent makes the
effects more concrete and relatable.  Her research on ``Weibo, which reportedly
employs around 1,000 people to monitor and censor user content and keep the
company in good standing with authorities,'' (MacKinnon) is a fantastic
representation of the possible overhead that could result from the passage of
the Protect IP Act.  MacKinnon does slip up a bit, however.  While she doesn't
actually state the slippery slope of the firewall, it is definitely implied by
her writing.  She implies that, if even a few websites with relatively
unimportant information are censored, then, soon, the government will be
censoring Facebook and Twitter and preventing revolutions like those in Egypt
earlier this year.  The argument also fails to provide a counter-argument, which
definitely hurts the authors ethos and makes her appear very one-sides.  While
the argument wasn't flawless, I felt like it provided the right mix between
pathos, in that it could limit freedom of speech and infringe on one's rights,
and logos, in that it would inflict significant and unnecessary overhead on
American corporations, to make an effective argument.

% -- SECTION 3 --
% Reflect on connection to Essay 3
This article has helped to show me that causal argument don't necessarily have
to be about something in the past, in which you are only arguing about things
that can be more thoroughly backed up.  Instead, they can argue about things
that haven't even happened yet, and perhaps change the outcome of that thing.  I
found the use of China as an example, combined with the title inspired by the
Great Wall of China, to be the most effective piece of the argument.  It also
provides extra pathos as many Americans may see China as opposition, and
wouldn't necessarily want to follow in their footsteps.


\begin{workscited}

\bibent MacKinnon, Rebecca. ``Stop the Great Firewall of America'' \emph{The New
York Times}. NYTimes, 15 Nov. 2011. Web. 19 Nov. 2011. 

\end{workscited}
\end{mla}
\end{document}
