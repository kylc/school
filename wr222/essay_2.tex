\documentclass[12pt,letterpaper]{article}

\usepackage{ifpdf}
\usepackage{mla}
\usepackage[utf8]{inputenc}

\begin{document}
\begin{mla}{Kyle}{Cesare}{Roush}{WR222 Section 001}{November 11, 2011}{
Corporations are not People}
% TODO: Capitalization correct here?  "Are Not"?

% -- SECTION 1 --
% Introduction
% Catchy opening
%  - Corporation never had some rights
%  - Now, Citizens United, they have...
% What is at stake?
% Who are the stakeholders?
A corporation has never in history been considered a citizen, had voting rights,
or had the same rights as the people of this country.  Now, however, even this
most fundamental definition is changing.  The Supreme Court Citizens case of
United v. Federal Election Commision now allows corporations to spend unlimited
amounts of money in support of or in attack of political candidates (Overby).
These sweeping changes were finalized in early 2010 in a controversial 5-4
ruling, and have ``sharply altered the debate over regulating political money.''
(Overby) Corporations should not be considered people in that they should not be
given the same political rights as people.  Others disagree: Justice Roberts
and Justice Alito, for instance, believe that prohibiting corporations' use of
money for political purposes is akin to censorship (Citizens United v. FEC).

% -- SECTION 2 --
% Body (Criteria/Reasons, Evidence, Opposition)
In order to properly evaluate whether or not corporations are people, we must
first decide what defines a person, and then use that definition as a basis of
evaluation of corporations.  On the most basic level, a person is defined as
``the body and clothing of a human being person.'' (Dictionary.com) We can see
that this definition expressly requires the entity be a living, breathing human
being.  Another take on the criteria is asking what an entity needs to ``think''
to be considered a person; this is akin to the reasoning behind not considering
a cat or another animal a person.  Obviously, a person needs to be sentient and
have thoughts of their own.  A final, and, arguably the most important, criteria
of a person is that it has emotion.  It can look at a situation and, rather than
simply asking ``Will this benefit me?'' it can ask ``How will this benefit/harm
others?''
% TODO: Finish definition of a person TODO: Dictionary.com citation correct?


% -- SECTION 3 --
% Evidence
Now, we ask ourselves if a corporation fulfills the requirements laid out above.
Quite obviously, a corporation is not a living, breathing entity, and so fails
the first test.  But what does a corporation think?  One may be tempted to say
that it thinks the sum of the ideas of its individuals or that it thinks
whatever the person in charge of it thinks.  However, in reality, a corporation
thinks nothing.  It has no brain or emotional repository with which to assign
thoughts to.  It is merely lead along by some controller, be it the employees,
the CEO, or the board members.  Therefore, a corporation does not think; rather,
it is thought for.

For the final question, does a corporation have emotion?  In 1982 a number of
people died from poisoned Tylenol, and Johnson \& Johnson was left with a
dilemma.  They could go public with the truth and admit that they had botched
the manufacturing process, or they could keep it quiet and hope nobody realized
what was happening.  The company decided to release a press statement with the
true events.  Now, without analyzing the situation, one might assume that this
was out of a desire to be truthful; perhaps, a showing of its emotional side.
However, the true motivation behind it's actions becomes apparent with further
inspection: profit.  The company, perhaps successfully, created the illusion of
an emotional being where there wasn't one; it emulated a person, but, still, it
was not one (Gutting).

Justice Stevens, one of the main opponents of giving corporations personhood,
creates a compelling argument:

\begin{quote}
In the context of election to public office, the distinction between corporate
and human speakers is significant.  Although they make enormous contributions to
our society, corporations are not actually members of it. They cannot vote or
run for office. Because they may be man aged and controlled by nonresidents,
their interests may conflict in fundamental respects with the interests of
eligible voters. The financial resources, legal structure, and instrumental
orientation of corporations raise legitimate concerns about their role in the
electoral process.  Our lawmakers have a compelling constitutional basis, if not
also a democratic duty, to take measures designed to guard against the
potentially deleterious effects of corporate spending in local and national
races (Citizens United v. FEC).
\end{quote}


% -- SECTION 4 --
% Balance with Opposition
Obviously, not everyone agrees that corporations should not be considered
people.  One notable exception is Supreme Court Justice Roberts, who published a
concurring Supreme Court ruling.  Roberts states that, in preventing unlimited
political contributions, we are ``[embracing] a theory of the First Amendment
that would allow censorship not only of television and radio broadcasts, but of
pamphlets, posters, and the Internet, and virtually any other medium that
corporations and unions might find useful in expressing their view on matters of
public concern.'' (Citizens United v. FEC)  There are two blaring fallacies in
this argument.  He's walking on a slippery slope in stating that, given this
moderate position on campaign finances, all of America will soon descend into a
censored dystopia.  In reality, corporations are readily able to express their
views through a number of mediums.  The only restriction is that they cannot
spend money, without going through a PAC, directly on attacking or supporting a
political candidate (Citizens United v.  FEC).  And even then, in the court case
in question, that was only within thirty days before the election.  Corporations
are given plenty of opportunity to voice opinions.

% -- SECTION 5 --
% Conclusion
In conclusion, corporations are not people.  They do not display the
characteristics required to be a person: they are not alive, they are not
sentient, and they do not have any emotion beyond the drive for increased
performance and profits.  There are some qualified people who disagree with the
argument laid out above, but there are also extremely qualified people who agree
that corporations are not people, and so shouldn't be given the same rights as
people.  The reality is that corporations aren't people, so why should they be
given the same rights as you and me?

\begin{workscited}

% TODO: Alphabetize these

% Supreme Court Decision (Citizens United v. FEC)
\bibent Citizens United v. Federal Election Commission.  No. 08-205.  Supreme
Ct. of the US.  21 Jan. 2010.  Print.

\bibent Gutting, Gary. ``Corporations, People and Truth.'' \emph{The New York
Times}.  NYTimes, 12 Oct. 2011. Web. 5 Nov. 2011. 

\bibent Overby, Peter. ``A Year Later, Citizens United Reshapes Politics.``
\emph{National Public Radio}. N.p., 21 Jan. 2011. Web. 5 Nov. 2011.

% Definition of a person
\bibent ``Person.'' \emph{Merriam-Webster's Dictionary of Law}. Merriam-Webster,
Inc. 06 Nov. 2011. $<$Dictionary.com
http://dictionary.reference.com/browse/person$>$.

% Overview from the time (Roberts)
\bibent ``Supreme Court Lifts Campaign Spending Li.'' Host Rebecca Roberts.
\emph{Talk of the Nation}. National Public Radio. NPR, 21 Jan. 2010. Web. 3 Nov.
2011. 

\end{workscited}
\end{mla}
\end{document}
