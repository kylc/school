\documentclass[12pt,letterpaper]{article}

\usepackage{ifpdf}
\usepackage{mla}
\usepackage[utf8]{inputenc}

\begin{document}
\begin{mla}{Kyle}{Cesare}{Roush}{WR222 Section 001}{October 14, 2011}{
Occupy Wall Street}

% -- SECTION 1 --
% Introduction
% - DYK top 1% hold 1/3 of all wealth in the U.S. (Rampell 1)
% - This is result of survey, and some people protesting it
% - Introduce OWS, when they started, who they are, what they want
% - Thesis: these protests for good cause, etc...
% TODO: OWS citation correct format?
Did you know that, in the United States of America, the top one percent of the
population holds more than one-third of all the wealth in the country (Rampell
1)?  In the past few weeks, a band of protesters on Wall Street and around the
country have formed in peaceful alliance against those behind this staggering
statistic, and the supposed havoc they have caused.  This series of protests has
come to be known as Occupy Wall Street, and is gaining momentum quickly.  The
protests have been in the news for almost a month now, and show no sign of
giving in to political pressures to disband.  The group's slogan has come to be
known as ``We are the 99\%,'' referring to the idea that 99\% percent of the
people in America are tired of the greed of the other 1\% (Occupy Wall Street).
It is important that the Occupy Wall Street protests continue, because the facts
behind their claims are valid and, if no action is taken, the corruption of the
upper-class will simply continue.

% -- SECTION 2 --
% Criteria of Evaluation
% - Qualitative
%   - Is the evidence sound (99% thing)?
%   - Who is backing it?
%   - Does it have a sound plan in place?
% - Quantitative
%   - Is the country really in the wrong direction?
%   - How has it been received publicly?
The Occupy Wall Street protests will be evaluated on a set of criteria.  First,
is their claim that 1\% of the population controls so much of the wealth really
valid?  Second, how much backing does this movement have?  Is it merely a small
group of protesters in New York receiving an overblown amount of news coverage,
or is it as humongous as we are lead to believe?  Next, does the group have a
sound cause and goal, or is the group too disparate to come together on a few
concentrated points?  Following this, how has the public reception been?
Gauging the public's response to the protests is a good way to determine whether
or not the group will be successful.

% -- SECTION 3 --
% Evidence
% - Evidence sound?
%   - Top 1% hold over 30% of wealth in U.S. (Rampell 1)
% - Who is backing?
% - Sound plan in place?
% - Is country really wrong direction?
% - How is reception?
The Occupy Wall Street group claims that 1\% of the U.S. population control more
than their fair share of wealth.  Studies featured in the Oxford University
Press show that the top 1\% control about 30\% of the wealth in the United
States, and receive about 21\% of the income (Rampell 1).  In a terrifying
disparity, the bottom 80\% of the population controls only about 15\% of the
country's wealth (1).  From these statistics, we gather that just three million
people control twice as much wealth as another 240 million.  And this problem
isn't going away; rather, it's getting worse.  Since the early 1980s, the bottom
80\% have been losing ground as the top percent added to its wealth.  We can see
that, quite obviously, the wealth distribution in the United States is horribly
lopsided, so Occupy Wall Street seems to be correct in these claims.

However, even with every fact in the world, the movement would be no good
without a substantial following.  So, how large of a following does the group
have?  According to Erik Eckholm, a writer for the New York Times, similar
protests have been springing up all around the country.  The movement has gained
an immense following, ``with protesters camped out in Los Angeles near City
Hall, assembled before the Federal Reserve Bank in Chicago and marching through
downtown Boston to rally against corporate greed'' (Eckholm 1).  Protests have
also sprung up in Tennessee, Baltimore, Texas, and even Hilo, Hawaii (1).  It
seems that the group has gathered a substantial following, and so fulfils
another category of its evaluation.

Given that the claims seem valid and it has a large following, does it have a
real stated goal?  The Occupy Wall Street website states that ``We need to
retake the freedom that has been stolen from the people, altogether'' (Occupy
Wall Street).  Their mission statement includes things like ``If you agree that
freedom for some is not the same as freedom for all, and that freedom for all is
the only true freedom, then you might be one of us'' and ``If you agree that
power is not right, that life trumps property, then you might be one of us''
(Occupy Wall Street).  These are not just hollow goals, however.  The
organization provides a call to action, asking people to ``[form] assemblies in
every city, every public square, every township,'' (Occupy Wall Street) and,
most importantly, to stand up for their freedom.  With that, we see that the
group has a clearly defined set of goals and a call to action to get people
involved.

Finally, how has the public taken to the movement?  Public reception seems to
have been, overall, quite good.  This has been boosted by lots of media coverage
and some police brutality (Goldstein 1).  In the case that received the most
publicity, a police officer pepper sprayed four women for no apparent reason,
and the encounter was captured on video (1).  Social media and videos are
proving very important for swaying the public's opinion.

% -- SECTION 4 --
% Balance
% - Organization is flawed
Unfortunately, not everything is going to plan.  One major problem is the lack
of support from politicians.  While some Democrats have said that they back the
movement, some ``see the prospect of the protesters' pushing the party
dangerously to the left -- just as the Tea Party has often pushed Republicans
farther to the right and made for intraparty run-ins'' (Lichtblau 1).  Lichtblau
goes on to state how important recognition from seniority of the Democratic
Party would be, just as it was for the Tea Party a few years ago.  Without this
important backing from politicians, a gaping hole is left in the protester's
ability to execute the changes they want effectively.

Another possible problem is public support in the future.  Will it continue to
be as strong as it has proven to be over the past few weeks, or will it begin to
taper off as it fails to produce any results.  Only time will tell the outcome,
but the uncertainty is certainly there.

% -- SECTION 5 --
% Conclusion
In conclusion, the evidence clearly shows that the Occupy Wall Street movement
deserves our attention and involvement.  The facts backing their argument are
easily verified, the group is gaining a large following, they have a clearly
defined goal, and a means of reaching it.  So, given the evidence, are you part
of the 99\%?

\begin{workscited}

% TODO: Alphabetize these

% Size of the protests, spread
% http://www.nytimes.com/2011/10/04/us/anti-wall-street-protests-spread-to-other-cities.html
\bibent Echkholm, Erik, and Timothy Williams. ``Anti-Wall Street Protests
Spreading to Cities Large and Small.'' \emph{The New York Times}. NYTimes, 3
Oct. 2011.  Web. 13 Oct. 2011.

% Pepper spray women
% http://www.nytimes.com/2011/09/26/nyregion/videos-show-police-using-pepper-spray-at-protest.html?scp=2
\bibent Goldstein, Joseph. ``Videos Show Police Using Pepper Spray at Protest on
the Financial System.'' \emph{The New York Times}. NYTimes, 25 Sept. 2011. Web.
17 Oct. 2011.

% Unions joining the protests
% http://www.nytimes.com/2011/10/06/nyregion/major-unions-join-occupy-wall-street-protest.html
\bibent Greenhouse, Steven, and Cara Buckley. ``Seeking Energy, Unions Join
Protest Against Wall Street.'' \emph{The New York Times}. NYTimes, 5 Oct. 2011.
Web. 13 Oct. 2011.

% Politicians backing the protests
% http://www.nytimes.com/2011/10/11/us/politics/wall-street-protests-gain-support-from-leading-democrats.html?pagewanted=all
\bibent Lichtblau, Eric. ``Democrats Try Wary Embrace of the Protests.''
\emph{The New York Times}. NYTimes, 10 Oct. 2011. Web. 16 Oct. 2011. 

% Unofficial website of OWS
% http://www.occupywallst.org/
\bibent Occupy Wall Street. \emph{Occupy Wall Street}. N.p., n.d. Web. 13 Oct.
2011.  $<$http://www.occupywallst.org/$>$.

% Wealth in the U.S. stats
% http://economix.blogs.nytimes.com/2011/03/30/inequality-is-most-extreme-in-wealth-not-income/
\bibent Rampell, Catherine. ``Inequality Is Most Extreme in Wealth, Not
Income.'' \emph{The New York Times}. NYTimes, 30 Mar. 2011. Web. 11 Oct. 2011. 

\end{workscited}
\end{mla}
\end{document}
