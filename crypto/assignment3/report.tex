\documentclass[12pt,letterpaper]{article}

\usepackage{fullpage}
\usepackage{xspace,graphicx,amsmath,amssymb,xcolor}

\newcommand{\pct}{\mathbin{\%}}

\newcommand{\K}{\mathcal{K}}
\newcommand{\M}{\mathcal{M}}
\newcommand{\C}{\mathcal{C}}
\newcommand{\Z}{\mathbb{Z}}

\newcommand{\Enc}{\text{\sf Enc}}
\newcommand{\Dec}{\text{\sf Dec}}
\newcommand{\KeyGen}{\text{\sf KeyGen}}

% fancy script L
\usepackage[mathscr]{euscript}
\renewcommand{\L}{\ensuremath{\mathscr{L}}\xspace}
\newcommand{\lib}[1]{\ensuremath{\L_{\textsf{#1}}}\xspace}

\newcommand{\myterm}[1]{\ensuremath{\text{#1}}\xspace}
\newcommand{\bias}{\myterm{bias}}
\newcommand{\link}{\diamond}
\newcommand{\subname}[1]{\ensuremath{\textsc{#1}}\xspace}


%%% code boxes

\newcommand{\fcodebox}[1]{%
    \framebox{\codebox{#1}}%
}
\newcommand{\hlcodebox}[1]{%
    \fcolorbox{black}{myyellow}{\codebox{#1}}%
}


\usepackage{varwidth}

\newcommand{\codebox}[1]{%
        \begin{varwidth}{\linewidth}%
        \begin{tabbing}%
            ~~~\=\quad\=\quad\=\quad\=\kill
            #1
        \end{tabbing}%
        \end{varwidth}%
}


\definecolor{myyellow}{HTML}{F5F589}

\newcommand{\mathhighlight}[1]{\basehighlight{$#1$}}
\newcommand{\highlightline}[1]{%\raisebox{0pt}[-\fboxsep][-\fboxsep]{
    \hspace*{-\fboxsep}\basehighlight{#1}%
%}
}
\newcommand{\basehighlight}[1]{\colorbox{myyellow}{#1}}

\pagestyle{empty}

\begin{document}

Kyle Cesare \hfill
\today \hfill

{\center\textbf{Assignment 3} \\}

\begin{enumerate}
  \item To show that $F'(k, r)$ is a secure PRF we must show that for two
    libraries, \lib{prf-real} and \lib{prf-rand}, there is no program that can
    distinguish between them. To do that, we will make small incremental changes
    to \lib{prf-real} that, by previously proven security definitions, do not
    change the security of the library. In the end, we will show that through
    these changes we can transform \lib{prf-real} into \lib{prf-rand}. By
    showing this, we demonstrate that \lib{prf-real} leaks no information about
    the plaintext.

    First, we will write out \lib{prf-real} linked to some program $A$ as
    defined by the problem.
    \[
      A
      \link
      \fcodebox{
        $k \gets \{ 0, 1 \}^n$ \\
        \underline{$\subname{challenge}(r)$:} \\
        \> $p := F(k, r)$ \\
        \> $x := G(p)$ \\
        \> return $x$
      }
    \]

    The next step is to factor out the call to $F$ as another library.
    \[
      A
      \link
      \fcodebox{
        \underline{$\subname{challenge}(r)$:} \\
        \> $p := \subname{query'}(r)$ \\
        \> $x := G(p)$ \\
        \> return $x$
      }
      \link
      \fcodebox{
        $k \gets \{ 0, 1 \}^n$ \\
        \underline{$\subname{query}(r)$:} \\
        \> $p := F(k, r)$ \\
        \> return $p$
      }
    \]

    Based on the security of the PRF $F$, we can replace $\lib{prf-real}^F$ with
    a random implementation $\lib{prf-rand}^F$.
    \[
      A
      \link
      \fcodebox{
        \underline{$\subname{challenge}(r)$:} \\
        \> $p := \subname{query'}(r)$ \\
        \> $x := G(p)$ \\
        \> return $x$
      }
      \link
      \fcodebox{
        $T$ := empty \\
        \underline{$\subname{query}(r)$:} \\
        \> if $T[r]$ undefined \\
        \>\> $T[r] \gets \{ 0, 1 \}^n$ \\
        \> return $T[r]$
      }
    \]

    Now, if we ensure we never choose an already-used $r$ then we can simplify
    \subname{query} to true randomness.
    \[
      A
      \link
      \fcodebox{
        $R$ := empty \\
        \underline{$\subname{challenge}(r)$:} \\
        \> $r \gets \{ 0, 1 \}^n \setminus R$ \\
        \> $R$ := $R \cup \{ r \}$ \\
        \> $p := \subname{query'}(r)$ \\
        \> $x := G(p)$ \\
        \> return $x$
      }
      \link
      \fcodebox{
        \underline{$\subname{query}(r)$:} \\
        \> $x \gets \{ 0, 1 \}^n$ \\
        \> return $x$
      }
    \]

    We can then combine the library we pulled out.
    \[
      A
      \link
      \fcodebox{
        $R$ := empty \\
        \underline{$\subname{challenge}(r)$:} \\
        \> $r \gets \{ 0, 1 \}^n \setminus R$ \\
        \> $R$ := $R \cup \{ r \}$ \\
        \> $p \gets \{ 0, 1 \}^n$ \\
        \> $x$ := $G(p)$ \\
        \> return $x$
      }
    \]

    Based on the security of $G$ we can apply almost the same operation to its
    call.
    \[
      A
      \link
      \fcodebox{
        $R$ := empty \\
        \underline{$\subname{challenge}(r)$:} \\
        \> $r \gets \{ 0, 1 \}^n \setminus R$ \\
        \> $R$ := $R \cup \{ r \}$ \\
        \> $x \gets \{ 0, 1 \}^n$ \\
        \> return $x$
      }
    \]

    And we can now remove the unused code.
    \[
      A
      \link
      \fcodebox{
        \underline{$\subname{challenge}(r)$:} \\
        \> $x \gets \{ 0, 1 \}^n$ \\
        \> return $x$
      }
    \]

    Obviously, this library does not leak information about the plaintext $r$
    because the plaintext is completely ignored!

  \item To show that $F'(k, r) = F(k, r) \oplus F(k, \bar{r})$ is not a PRF we
    must provide a program that can distinguish $F'$ from true randomness
    non-negligibly. Such a program is shown below:

    \[
      \fcodebox{
        \underline{$\subname{A}$:} \\
        \> $x \gets \{ 0 \}^n$ \\
        \> $c_1 = \subname{query}(x)$ \\
        \> $c_2 = \subname{query}(\bar{x})$ \\
        \> return $c_1 == c_2$
      }
    \]

    When linked to $\lib{prf-real}^F$ this program will return $1$ with
    probability $1$. When linked to $\lib{prf-rand}^F$ it will return $1$ with
    probability $\frac{1}{2^n}$. Because this difference is non-negligible, a
    polynomial-time program is able to distinguish it from true randomness.
    Therefore, the PRF is not secure Therefore, the PRF is not secure

  \item We must show that any 2-round Feistel network cannot be a secure PRF. To
    do this, we will create a distinguisher that demonstrates a non-negligible
    bias between true randomness and the Feistel network $F$.

    Our distinguisher will exploit the fact that a 2-round Feistel network
    $F(a, b) = (f(b) \oplus a, f[f(b) \oplus a) \oplus b]$. Because of this, if
    we call $F$ once with the string $x$ then we can call it again with the left
    and right halves of the output reversed. We can then make use of the
    property mentioned above to check the second half of the second output. If
    the program is linked to the Feistel network then it will always be equal to
    $0$.

    \[
      \fcodebox{
        \underline{$\subname{A}$:} \\
        \> $x \gets \{ 0 \}^n$ \\
        \> $L_1, R_1 = F(x_l, x_r)$ \\
        \> $L_2, R_2 = F(R_l, L_r)$ \\
        \> return $L_2 == 0$
      }
    \]

  \item We will show that the encryption scheme does not have CPA-security by
    providing a distinguishing program $A$ that shows different biases when
    linked to $\lib{prp-real}^F$ and $\lib{prp-rand}^F$.

    \[
      \fcodebox{
        \underline{$\subname{A}$:} \\
        \> $x \gets \{ 0, 1 \}^n$ \\
        \> $y \gets \{ 0, 1 \}^n$ \\
        \> $a = \subname{challenge}(x, x)$ \\
        \> $u = a_l \oplus a_r$ \\
        \> $b = \subname{challenge}(x, y)$ \\
        \> $v = b_l \oplus b_r$ \\
        \> return $u == v$
      }
    \]

    The bias when linked to $\lib{prp-real}^F$ is $1$, but when linked to
    $\lib{prp-rand}^F$ it is $\frac{1}{2^n}$.

\end{enumerate}

\end{document}
