\documentclass[12pt,letterpaper]{article}

\usepackage{fullpage}
\usepackage{xspace,graphicx,amsmath,amssymb,xcolor}
\usepackage[T1]{fontenc}
\usepackage{numprint}

\npthousandsep{ }

\newcommand{\pct}{\mathbin{\%}}

\newcommand{\K}{\mathcal{K}}
\newcommand{\M}{\mathcal{M}}
\newcommand{\C}{\mathcal{C}}
\newcommand{\Z}{\mathbb{Z}}

\newcommand{\Enc}{\text{\sf Enc}}
\newcommand{\Dec}{\text{\sf Dec}}
\newcommand{\KeyGen}{\text{\sf KeyGen}}

% fancy script L
\usepackage[mathscr]{euscript}
\renewcommand{\L}{\ensuremath{\mathscr{L}}\xspace}
\newcommand{\lib}[1]{\ensuremath{\L_{\textsf{#1}}}\xspace}

\newcommand{\myterm}[1]{\ensuremath{\text{#1}}\xspace}
\newcommand{\bias}{\myterm{bias}}
\newcommand{\link}{\diamond}
\newcommand{\subname}[1]{\ensuremath{\textsc{#1}}\xspace}


%%% code boxes

\newcommand{\fcodebox}[1]{%
    \framebox{\codebox{#1}}%
}
\newcommand{\hlcodebox}[1]{%
    \fcolorbox{black}{myyellow}{\codebox{#1}}%
}


\usepackage{varwidth}

\newcommand{\codebox}[1]{%
        \begin{varwidth}{\linewidth}%
        \begin{tabbing}%
            ~~~\=\quad\=\quad\=\quad\=\kill
            #1
        \end{tabbing}%
        \end{varwidth}%
}


\definecolor{myyellow}{HTML}{F5F589}

\newcommand{\mathhighlight}[1]{\basehighlight{$#1$}}
\newcommand{\highlightline}[1]{%\raisebox{0pt}[-\fboxsep][-\fboxsep]{
    \hspace*{-\fboxsep}\basehighlight{#1}%
%}
}
\newcommand{\basehighlight}[1]{\colorbox{myyellow}{#1}}

\pagestyle{empty}

\begin{document}

Kyle Cesare \hfill
\today \hfill

{\center\textbf{Assignment 7} \\}

\begin{enumerate}
  \item Given the parameters $p$, $g$, $A$, and $B$, the shared key $g^{ab} =
    19266619797362$. We are able to compute this only because the we used
    $\mathbb{G} = \Z_p^*$. The prime $p$ is not a ``safe'' prime, meaning that
    we are able to easily factor $p-1$. Because of this, we can use methods like
    the Pohlig-Hellman algorithm, Shanks baby-step/giant-step algorithm, and the
    Pollard rho method. For this calculation, I used the Pari/GP \texttt{znlog}
    function.

  \item Given an ElGamal ciphertext $(B, C)$ of an unknown plaintext $M \in
    \mathbb{G}$, the generator $g$, and the public key $A$, we can construct
    another ciphertext $(B', C')$ that decrypts to the same $M$.

    To be decrypted to the same message, we must change $B$ and $C$ in such a
    way that the private key calculated from $B$ will remain the inverse. To do
    this, we can modify the message to $(B * g, C * A)$.

    Under $\Dec$, $K = B^a$ will be computed. This will now become $K' = (B *
    g)^a = g^{(b+1)^a} = g^{ab + a}$. The cipertext $C'$ is now $C' = M *
    A^{b+1} = M * g^{a^{b+1}} = M * g^{ab + a}$.

    Now, $C' * K^{\prime-1} = M$, so the resulting plaintext has remain
    unchanged despite a change in the ciphertext.

  \item Any CPA-secure public key encryption scheme implies secure key
    agreement.

    Alice and Bob want to securely share some random data $r$ without an
    observer learning anything about $r$. To do this, Alice computes $r$
    randomly. She then encrypts the data under Bob's public key $e$ and
    transmits the ciphertext to Bob. Bob then decodes the data using his private
    key $d$.

    We can prove the KA security of this algorithm using the CPA security of
    RSA. We define two libraries, one which encrypts the actual key and returns
    the transcript of the conversation (only the encrypted key), and the other
    which returns the transcript of the actual key and a different random key.

    \begin{minipage}{.5\textwidth}
      $\lib{rsa-ka-real}^\Sigma$:
        \fcodebox{
          $(pk, sk) \gets \Sigma.\KeyGen$ \\ \\

          \underline{\subname{getpk():}} \\
          \> return $pk$ \\ \\

          \underline{\subname{query()}} \\
          \> $r \gets \{ 0, 1 \}^n$ \\
          \> $c := \Sigma.\Enc(pk, r)$ \\
          \> return $(c, r)$
        }
    \end{minipage}
    \begin{minipage}{.5\textwidth}
      $\lib{rsa-ka-rand}^\Sigma$:
        \fcodebox{
          $(pk, sk) \gets \Sigma.\KeyGen$ \\ \\

          \underline{\subname{getpk():}} \\
          \> return $pk$ \\ \\

          \underline{\subname{query()}} \\
          \> $r \gets \{ 0, 1 \}^n$ \\
          \> $c := \Sigma.\Enc(pk, r)$ \\
          \> $r^\prime \gets \{ 0, 1 \}^n$ \\
          \> return $(c, r')$
        }
    \end{minipage}

    Based on RSA CPA security, we can say that these two libraries are
    indistinguishable. Therefore, there is a negligible bias between the two
    libraries, so the scheme is KA secure.
\end{enumerate}

\end{document}
