\documentclass[12pt,letterpaper]{article}

\usepackage{fullpage}
\usepackage{xspace,graphicx,amsmath,amssymb,xcolor}

\newcommand{\pct}{\mathbin{\%}}

\newcommand{\K}{\mathcal{K}}
\newcommand{\M}{\mathcal{M}}
\newcommand{\C}{\mathcal{C}}
\newcommand{\Z}{\mathbb{Z}}

\newcommand{\Enc}{\text{\sf Enc}}
\newcommand{\Dec}{\text{\sf Dec}}
\newcommand{\KeyGen}{\text{\sf KeyGen}}

% fancy script L
\usepackage[mathscr]{euscript}
\renewcommand{\L}{\ensuremath{\mathscr{L}}\xspace}
\newcommand{\lib}[1]{\ensuremath{\L_{\textsf{#1}}}\xspace}

\newcommand{\myterm}[1]{\ensuremath{\text{#1}}\xspace}
\newcommand{\bias}{\myterm{bias}}
\newcommand{\link}{\diamond}
\newcommand{\subname}[1]{\ensuremath{\textsc{#1}}\xspace}


%%% code boxes

\newcommand{\fcodebox}[1]{%
    \framebox{\codebox{#1}}%
}
\newcommand{\hlcodebox}[1]{%
    \fcolorbox{black}{myyellow}{\codebox{#1}}%
}


\usepackage{varwidth}

\newcommand{\codebox}[1]{%
        \begin{varwidth}{\linewidth}%
        \begin{tabbing}%
            ~~~\=\quad\=\quad\=\quad\=\kill
            #1
        \end{tabbing}%
        \end{varwidth}%
}


\definecolor{myyellow}{HTML}{F5F589}

\newcommand{\mathhighlight}[1]{\basehighlight{$#1$}}
\newcommand{\highlightline}[1]{%\raisebox{0pt}[-\fboxsep][-\fboxsep]{
    \hspace*{-\fboxsep}\basehighlight{#1}%
%}
}
\newcommand{\basehighlight}[1]{\colorbox{myyellow}{#1}}

\pagestyle{empty}

\begin{document}

Kyle Cesare \hfill
\today \hfill

{\center\textbf{Assignment 2} \\}

\begin{enumerate}
  \item \begin{enumerate}
    \item $\frac{1}{2^{n/2}}$ is negligible.
    \item $\frac{1}{2^{\log n^2}}$ is not negligible because it can be reduced
      to $\frac{1}{n^2}$.
    \item $\frac{1}{n^{\log n}}$ is not negligible.
    \item $\frac{1}{2^{(\log n)^2}}$ is not negligible.
    \item $\frac{1}{(\log n)^2}$ is not negligible because it goes to zero much
      slower than $\frac{1}{n}$.
    \item $\frac{1}{n^{1/n}}$ is not negligible for the same reason as (e).
    \item $\frac{1}{\sqrt{n}}$ is not negligible for the same reason as (e).
    \item $\frac{1}{2^{\sqrt{n}}}$ is negligible.
    \item $\frac{1}{n^{\sqrt{n}}}$ is negligible.
  \end{enumerate}

  \item \begin{enumerate}
      \item The bias is $\frac{1}{2^n}$.
      \item No, it does not. For $G$ to be a PRG the bias must only be
        negligible, rather than zero. The bias shown above is negligible, so $G$
        is still a PRG.
      \item If $G$ is not injective then multiple inputs could map to the same
        output. This would cause the bias to grow.
    \end{enumerate}


  \item To show that $H$ is secure based on the security of $G$, we must show
    that no program $A$ can be used to distinguish between our pseudorandom
    output generated by $H$ and actual random data. 
    % Step 1
    First, we begin by filling in the definition of $H$.
    \[
      A
      \link
      \fcodebox{
        \underline{$\subname{query}()$:} \\
        \> $s \gets \{ 0, 1 \}^n$ \\
        \> $x := G(s_{\textsf{left}})$ \\
        \> $y := G(s_{\textsf{right}})$ \\
        \> return $x \oplus y$
      }
    \]
    % Step 2
    We can then pull out the calls to $G$.
    \[
      A
      \link
      \fcodebox{
        \underline{$\subname{query}()$:} \\
        \> $x := \subname{query'}()$ \\
        \> $y := \subname{query'}()$ \\
        \> return $x \oplus y$
      }
      \link
      \fcodebox{
        \underline{$\subname{query'}()$:} \\
        \> $s \gets \{ 0, 1 \}^n$ \\
        \> return $G(s)$
      }
    \]
    % Step 3
    Because we know $G$ is secure, we can replace any calls to $G$ with actual
    random data of the stretched length.
    \[
      A
      \link
      \fcodebox{
        \underline{$\subname{query}()$:} \\
        \> $x := \subname{query'}()$ \\
        \> $y := \subname{query'}()$ \\
        \> return $x \oplus y$
      }
      \link
      \fcodebox{
        \underline{$\subname{query'}()$:} \\
        \> $s \gets \{ 0, 1 \}^{3n}$ \\
        \> return $s$
      }
    \]
    % Step 4
    Finally, we inline the function and see that, based on the security of OTP,
    $H$ is secure.
    \[
      A
      \link
      \fcodebox{
        \underline{$\subname{query}()$:} \\
        \> $x \gets \{ 0, 1 \}^{3n}$ \\
        \> $y \gets \{ 0, 1 \}^{3n}$ \\
        \> return $x \oplus y$
      }
    \]
  \item $H$ is not a PRG, because it can be distinguished from true randomness
    with the following program $A$:
    \[
      \fcodebox{
        \underline{$\subname{A}$:} \\
        \> $s \gets \{ 0, 1 \}^n$ \\
        \> $x_{\textsf{left}}, x_{\textsf{right}} = H(s)$ \\
        \> return $x_{\textsf{left}} == x_{\textsf{right}}$
      }
    \]
    This program will always output $1$ when linked to the bad implementation of
    $H$, but very rarely when linked to true randomness. Expressed exactly:

    \[
      \bias(A,\lib{prg-real},\lib{prg-rand}) = 1 - 2^{\frac{n}{2}}
    \]

    This is not negligible, so we have shown that $H$ is not a true PRG.
\end{enumerate}

\end{document}
