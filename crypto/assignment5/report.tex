\documentclass[12pt,letterpaper]{article}

\usepackage{fullpage}
\usepackage{xspace,graphicx,amsmath,amssymb,xcolor}
\usepackage[T1]{fontenc}

\newcommand{\pct}{\mathbin{\%}}

\newcommand{\K}{\mathcal{K}}
\newcommand{\M}{\mathcal{M}}
\newcommand{\C}{\mathcal{C}}
\newcommand{\Z}{\mathbb{Z}}

\newcommand{\Enc}{\text{\sf Enc}}
\newcommand{\Dec}{\text{\sf Dec}}
\newcommand{\KeyGen}{\text{\sf KeyGen}}

% fancy script L
\usepackage[mathscr]{euscript}
\renewcommand{\L}{\ensuremath{\mathscr{L}}\xspace}
\newcommand{\lib}[1]{\ensuremath{\L_{\textsf{#1}}}\xspace}

\newcommand{\myterm}[1]{\ensuremath{\text{#1}}\xspace}
\newcommand{\bias}{\myterm{bias}}
\newcommand{\link}{\diamond}
\newcommand{\subname}[1]{\ensuremath{\textsc{#1}}\xspace}


%%% code boxes

\newcommand{\fcodebox}[1]{%
    \framebox{\codebox{#1}}%
}
\newcommand{\hlcodebox}[1]{%
    \fcolorbox{black}{myyellow}{\codebox{#1}}%
}


\usepackage{varwidth}

\newcommand{\codebox}[1]{%
        \begin{varwidth}{\linewidth}%
        \begin{tabbing}%
            ~~~\=\quad\=\quad\=\quad\=\kill
            #1
        \end{tabbing}%
        \end{varwidth}%
}


\definecolor{myyellow}{HTML}{F5F589}

\newcommand{\mathhighlight}[1]{\basehighlight{$#1$}}
\newcommand{\highlightline}[1]{%\raisebox{0pt}[-\fboxsep][-\fboxsep]{
    \hspace*{-\fboxsep}\basehighlight{#1}%
%}
}
\newcommand{\basehighlight}[1]{\colorbox{myyellow}{#1}}

\pagestyle{empty}

\begin{document}

Kyle Cesare \hfill
\today \hfill

{\center\textbf{Assignment 4} \\}

\begin{enumerate}
  \item To create a MAC forgery in this scheme, we only have to find a hash
    collision. When a user tries to log in, they will send their authentication
    code, $\text{MAC}(k, H(d))$, where $d = \text{username|date}$. The
    server will verify that the MAC decrypts to the correct username, and that
    the date part is less than the current date.

    So, to generate a forgery under this scheme we must find some username $u$
    and some date $d$ such that $H(u\text{|}20150227) = H(\text{mikero|}d)$. We
    are required to fix the date for the user we will be registering because the
    server will encode the data into the MAC, which we have no control over.
    However, we can pick a random date (as long is it is before $20150227$) for
    user \textit{mikero} because the server will only verify that it is less
    than the current date.

    Using the birthday bound, we can craft a program to find this collision in a
    reasonable amount of time. Because we have a 40-bit hash function, we expect
    to find a collision within $2^{20}$ attemps. So, we first generate a set of
    length $2^{20}$ filled with hashes for user \textit{mikero} with random
    dates. Next, we generate $2^{20}$ hashes for random usernames with the date
    $20150227$. We expect that there should be a collision between these two
    sets.

    Here is an example of such a collision:

    \begin{tabular}{ l  l  l  }
      Username & Date     & Hash \\ \hline
      mikero   & 14900767 & 2e85953aea \\
      jppem    & 20150227 & 2e85953aea \\
    \end{tabular}
    
  \item Given a collision under $H^{(t)}$, we would like to efficiently find a
    collision under $H$. We are given $a, b$ such that $H^{(t)}(a) =
    H^{(t)}(b)$. We can rewrite this as

    $$
      H(H^{(t-1)}(a)) = H(H^{(t-1)}(b))
    $$

    Based on this, we can see that $H^{(t-1)}(a)$ and $H^{(t-1)}(b)$ are a
    collision under a single application of $H$.

  \item To show that exponents in $\pmod{n}$ arithmetic can be reduced by
    $\phi(n)$, we must show that the $x^a \pmod{n}$ is equivalent to $(x^{a
    \bmod{\phi(n)}}) \pmod{n}$.
    
    Lagrange's theorem states that for any finite group $G$, the number of
    subelements divides the order of $G$. That is, the number of elements in the
    group of numbers coprime to $n$, or $\phi(n)$, divides $\mathbb{Z}_n$.
    Lagrange's theorem therefore states that $m^{\phi(n)} = 1$.

    We can rewrite the right-hand side of the original equation by following
    common exponentiation rules.
    $$
      (x^a)(x^{\phi(n)}) \pmod{n}
    $$

    And, as we have shown, $x^{\phi(n)} = 1$. So, this is equivalent to the
    left-hand side of the original equation.
    $$
      (x^a)(1) \pmod{n} = x^a \pmod{n}
    $$

    Therefore, we have shown that exponents in $\pmod{n}$ arithmetic can be
    reduced by $\phi(n)$.

\end{enumerate}

\end{document}
