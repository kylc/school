\documentclass[12pt,letterpaper]{article}

\usepackage{fullpage}
\usepackage{xspace,graphicx,amsmath,amssymb,xcolor}

\newcommand{\pct}{\mathbin{\%}}

\newcommand{\K}{\mathcal{K}}
\newcommand{\M}{\mathcal{M}}
\newcommand{\C}{\mathcal{C}}
\newcommand{\Z}{\mathbb{Z}}

\newcommand{\Enc}{\text{\sf Enc}}
\newcommand{\Dec}{\text{\sf Dec}}
\newcommand{\KeyGen}{\text{\sf KeyGen}}

% fancy script L
\usepackage[mathscr]{euscript}
\renewcommand{\L}{\ensuremath{\mathscr{L}}\xspace}
\newcommand{\lib}[1]{\ensuremath{\L_{\textsf{#1}}}\xspace}

\newcommand{\myterm}[1]{\ensuremath{\text{#1}}\xspace}
\newcommand{\bias}{\myterm{bias}}
\newcommand{\link}{\diamond}
\newcommand{\subname}[1]{\ensuremath{\textsc{#1}}\xspace}


%%% code boxes

\newcommand{\fcodebox}[1]{%
    \framebox{\codebox{#1}}%
}
\newcommand{\hlcodebox}[1]{%
    \fcolorbox{black}{myyellow}{\codebox{#1}}%
}


\usepackage{varwidth}

\newcommand{\codebox}[1]{%
        \begin{varwidth}{\linewidth}%
        \begin{tabbing}%
            ~~~\=\quad\=\quad\=\quad\=\kill
            #1
        \end{tabbing}%
        \end{varwidth}%
}


\definecolor{myyellow}{HTML}{F5F589}

\newcommand{\mathhighlight}[1]{\basehighlight{$#1$}}
\newcommand{\highlightline}[1]{%\raisebox{0pt}[-\fboxsep][-\fboxsep]{
    \hspace*{-\fboxsep}\basehighlight{#1}%
%}
}
\newcommand{\basehighlight}[1]{\colorbox{myyellow}{#1}}

\pagestyle{empty}

\begin{document}

Kyle Cesare \hfill
\today \hfill

{\center\textbf{Assignment 4} \\}

\begin{enumerate}
  \item
    \begin{enumerate}
      \item We must modify the CPA security definition for CBC mode encryption
        to allow calling programs to see the chosen IV before choosing a
        plaintext.

        \begin{minipage}{.5\textwidth}
          $\lib{cpa-L}^\Sigma$:
            \fcodebox{
              $r \gets \{ 0, 1 \}^n$ \\

              \underline{\subname{ivoracle():}} \\
              \> return $r$ \\ \\

              \underline{\subname{challenge($m_L, m_R \in \Sigma.\M$)}} \\
              \> $c := \Sigma.\Enc(k, r, m_L)$ \\
              \> $r \gets \{ 0, 1 \}^n$ \\
              \> return $c$
            }
        \end{minipage}
        \begin{minipage}{.5\textwidth}
          $\lib{cpa-L}^\Sigma$:
            \fcodebox{
              $r \gets \{ 0, 1 \}^n$ \\

              \underline{\subname{ivoracle():}} \\
              \> return $r$ \\ \\

              \underline{\subname{challenge($m_L, m_R \in \Sigma.\M$)}} \\
              \> $c := \Sigma.\Enc(k, r, m_R)$ \\
              \> $r \gets \{ 0, 1 \}^n$ \\
              \> return $c$
            }
        \end{minipage}
        
        We must then also modify the $\Enc$ function to accept the pre-chosen IV.
        The $\Dec$ function remains unchanged, as we still embed the IV into the
        output.

        \begin{minipage}{.5\textwidth}
          \fcodebox{
            \underline{$\Enc(x)_{cbc}^F(k, c_0, m_1 \cdots m_l)$:} \\
            \> for $i = 1 \text{ to } l$ \\
            \> \> $c_i := F(k, m_i \oplus c_{i-1})$ \\
            \> return $c_0c_1 \cdots c_l$
          }
        \end{minipage}
        \begin{minipage}{.5\textwidth}
          \fcodebox{
            \underline{$\Dec(x)_{cbc}^F(k, c_1 \cdots c_l)$:} \\
            \> for $i = 1 \text{ to } l$ \\
            \> \> $m_i := F^{-1}(k, c_i) \oplus c_{i-1}$ \\
            \> return $m_1 \cdots m_l$
          }
        \end{minipage}

      \item We will show that traditional CBC mode does satisfy this new CPA
        security definition by providing a distinguishing program $\subname{A}$
        that produced a bias when linked against $\lib{cpa-L}^\Sigma$ and
        $\lib{cpa-R}^\Sigma$. The distinguisher will make use of the fact that,
        by knowing the plaintext $m_1$ and the IV, it can determine the input to
        $F$. By knowing the input, CBC really becomes a deterministic encryption
        scheme, which we know we can distinguish.

        \[
          \fcodebox{
            \underline{\subname{A:}} \\
            \> $m_L := \{ 0 \}^n$ \\
            \> $m_R := \{ 1 \}^n$ \\
            \> $i := \subname{ivoracle}()$ \\
            \> $c_0 := \subname{challenge}(m_L \oplus i, m_R \oplus i)$ \\
            \> $c_1 := \subname{challenge}(m_L \oplus i, m_L \oplus i)$ \\
            \> return $c_0 == c_1$
          }
        \]

        This produces a bias of $1 - \frac{1}{2^n}$, which is not negligible.

    \end{enumerate}

  \item To show that this is modified-CPA\$ secure we must show that, given an
    adversary with access to a $\subname{challenge}(iv, m_L, m_R)$ function, the
    adversary cannot tell if $m_L$ is being encrypted or if $m_R$ is being
    encrypted. That is, the function returns pseudorandom ciphertexts. First,
    let's start out with the definition of the $\Enc$ function.
    \[
      A
      \link
      \fcodebox{
        $k \gets \{ 0, 1 \}^n$ \\
        \underline{$\subname{challenge}(iv, m_L, m_R)$:} \\
        \> $iv' := F(k_{\text{left}}, iv)$ \\
        \> $c := \text{CBC encryption of } m_L$ \\
        \>\> $\text{under key } k$ \\
        \>\> $\text{using initialization vector } iv'$ \\
        \> $\text{return } c$
      }
    \]

    Factor out the CBC encryption scheme based on its CPA\$ security.
    \[
      A
      \link
      \fcodebox{
        \underline{$\subname{challenge}(iv, m_L, m_R)$:} \\
        \> $iv' := F(k_{\text{left}}, iv)$ \\
        \> $c := \subname{challenge'}(iv', m_L)$ \\
        \> $\text{return } c$
      }
      \link
      \fcodebox{
        $k \gets \{ 0, 1 \}^n$ \\
        \underline{$\subname{challenge'}(iv, m_L, m_R)$:} \\
        \> $c := \text{CBC encryption of } m_L$ \\
        \>\> $\text{under key } k$ \\
        \>\> $\text{using initialization vector } iv$
      }
    \]

    Based on the security definition of $F$ and the fact that an IV cannot be
    repeated, we can replace $iv'$ with randomness.
    \[
      A
      \link
      \fcodebox{
        \underline{$\subname{challenge}(iv, m_L, m_R)$:} \\
        \> $c := \subname{challenge'}(m_L, m_R)$ \\
        \> $\text{return } c$
      }
      \link
      \fcodebox{
        $k \gets \{ 0, 1 \}^n$ \\
        \underline{$\subname{challenge'}(m_L, m_R)$:} \\
        \> $iv := \{ 0, 1 \}^n$ \\
        \> $c := \text{CBC encryption of } m_L$ \\
        \>\> $\text{under key } k$ \\
        \>\> $\text{using initialization vector } iv$
      }
    \]

    Based on the CPA\$ security of CBC encryption, replace the
    $\subname{challenge'}$ subroutine with a random ciphertext.
    \[
      A
      \link
      \fcodebox{
        \underline{$\subname{challenge}(iv, m_L, m_R)$:} \\
        \> $c := \subname{challenge'}(m_L, m_R)$ \\
        \> $\text{return } c$
      }
      \link
      \fcodebox{
        \underline{$\subname{challenge'}(m_L, m_R)$:} \\
        \> $\text{return } c \gets \mathcal C$
      }
    \]

    And now we can move everything back into the main library.
    \[
      A
      \link
      \fcodebox{
        \underline{$\subname{challenge}(iv, m_L, m_R)$:} \\
        \> $\text{return } c \gets \mathcal C$
      }
    \]

    So, we can see that this satisfied CPA\$ security by returning pseudorandom
    ciphertexts.

  \item To show that $\Sigma^2$ is not CCA secure, we must describe a
    distinguisher that produces a bias when linked to a library that encrypts
    $m_L$ compared to when it is linked to a library that encrypts $m_R$.
    Because we are in the CCA-security world, the distinguisher has the ability
    to decrypt arbitrary messages, as long as the messages were not previously
    passed into the challenge function.

    The intuition for this distinguisher is to realize that message data is
    duplicated in the ciphertext. Therefore, if we can jumble the ciphertext to
    produce another ciphertext that decrypts to the same message then we can
    call the decrypt function directly on this new ciphertext.
    
    Such a distinguisher is defined below:
    \[
      \fcodebox{
        \underline{$\subname{A}$:} \\
        \> $x \gets \{ 0 \}^n$ \\
        \> $y \gets \{ 1 \}^n$ \\
        \> $c_{\text{left}}, c_{\text{right}} = \subname{challenge}(x, y)$ \\
        \> $m = \subname{dec}(c_\text{right} || c_\text{left})$ \\
        \> return $m == x$
      }
    \]

    This program produces a bias of $1 - \frac{1}{2^n}$, which is not
    negligible, so we have shown that $\Sigma^2$ is not CCA secure.

\end{enumerate}

\end{document}
