\documentclass[12pt,letterpaper]{article}

\usepackage{fullpage}

\pagestyle{empty}

\begin{document}

Kyle Cesare \hfill
\today \hfill

{\center\textbf{Assignment 1} \\}

\begin{enumerate}
  \item \begin{enumerate}
    \item This encryption scheme is not perfectly secure. Shown below is a table
      of all 1-bit plaintexts encrypted with a 1-bit key. Notice that the
      distributions of ciphertexts are different depending on the plaintext. \\
      \begin{tabular}{ l | l l l }
        Enc($k$, $m$) & $k$ = 0 & 1 \\ \hline
        $m$ = 0       &       0 & 0 \\
              1       &       0 & 1 \\
      \end{tabular}
    \item This scheme is perfectly secure. The distribution of the ciphertexts
      is uniformly distributed over $\{ 0, 1, \ldots, 2^{n-1} \}$.
  \end{enumerate}

  \item This argument that one-time pad is not perfectly secret is flawed
    because it assumes that Eve is able to recognize when she has found the key
    $k$ that was used. This is impossible to do without context, as a single
    n-byte cipthertext $c$ can be transformed into any plaintext message $m \in
    M$. For example, the ciphertext of \texttt{HELLO} could be decrypted to any
    other five letter combination of alphabet letters solely based on the key
    chosen.

  \item Eve has now seen $c = m \oplus k$ and $c' = m' \oplus k$. Written
    another way, $k = c \oplus m$ and $k = c' \oplus m'$, so $c \oplus m = c'
    \oplus m'$. Therefore $c \oplus c' = m \oplus m'$. In other words, by XORing
    the two ciphertexts, Eve can learn the XOR of the two messages.

  \item For a $t$-out-of-$n$ secret sharing scheme to be secure, we require that
    for any party $|X| < t$ there is no information about the plaintext learned.
    In a 5-out-of-5 sharing scheme, for example, if a party of four knows that
    the fifth party's share is less than $k$ bits long, then there must be some
    information embedded in their shares that depend on the message. This
    violates the security principle of secret sharing, which states that for any
    deficient set $X \subseteq \{ 1, \ldots, n \}$ and $m, m' \in M$, the
    distributions $\{ S \gets \mbox{Share}(m); S|_X \}$ and $\{ S \gets
    \mbox{Share}(m); S|_X \}$ are identical. These distributions cannot be equal
    because there must be some information about the message encoded in their
    shares, which violates the security of the scheme.

  \item In this secret sharing problem we have three groups of three people
    each, and the secret can only be opened if at least 2-out-of-3 people from
    each committee are present. This can be broken into three 2-out-of-3
    problems and one 3-out-of-3 problem.

    That is, for each subcommittee, use a 3-out-of-3 secret sharing scheme to
    ensure that all three subcommittees must be present to unlock the secret.
    Then, within each subcommittee, use a 2-out-of-3 secret sharing scheme so
    that two of the three members of each subcommittee must be present to unlock
    their share of the main secret.
\end{enumerate}

\end{document}
