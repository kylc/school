\documentclass[11pt]{article}

\usepackage[top=1in,right=1in,left=1in,bottom=1in]{geometry}
\usepackage{hyperref}

\begin{document}

\begin{flushright}
  Kyle Cesare
\end{flushright}

{\center \textbf{Lab 5} \\}

\begin{enumerate}
  \item The two access points sending the most beacon frames have the SSIDs
    \texttt{30 Munroe St} and \texttt{linksys12}.

  \item \texttt{30 Munroe St} rebroadcasts at 0.102400 second intervals.
    \texttt{linksys12} rebroadcasts at the same rate.

  \item The source MAC address for beacon frames from the \texttt{30 Munroe
    St} AP is \texttt{00:16:b6:f7:1d:51}.

  \item The destination MAC address for beacon frames from the \texttt{30 Munroe
    St} AP is the broadcast address, \texttt{ff:ff:ff:ff:ff:ff}.

  \item The MAC BSS ID is the same as the source MAC address for this AP:
    \texttt{30 Munroe St}.

  \item The AP claims it supports the following data rates: 1Mbit/s, 2Mbit/s,
    5.5Mbit/s, 11Mbit/s.  It also claims to support the following extended data
    rates: 6Mbit/s, 9Mbit/s, 12Mbit/s, 18Mbit/s, 24Mbit/s, 36Mbit/s, 48Mbit/s,
    and 56Mbit/s.

  \item The BSS ID, which is the access point, is \texttt{00:16:b6:f7:1d:51}.
    The source address, the wireless host, is \texttt{00:13:02:d1:b6:4f}.  The
    destination address, the first-hop router, it \texttt{00:16:b6:f4:eb:a8}.
    The source IP address is \texttt{192.168.1.109}, and the destination IP
    address is \texttt{128.119.245.12}.  This destination IP address does not
    correspond to any hardware address, because it is not within the local
    subnet.

  \item In the next \texttt{SYNACK}, the destination and source MAC address are
    swapped from the previous \texttt{SYN} request.  Now, the wireless host is
    the destination and the first-hop router is the source.  The AP is still the
    BSS ID.  Again, the sender MAC address does not correspond to the source IP
    address, as the two hosts are not on the same local subnet.

  \item To disassociate, the wireless host first sends a DHCP release request to
    the AP, followed by an 802.11 Deauthentication frame.  From the 802.11
    specification, one would expect to see a Disassociation frame following the
    Deauthentication frame, but it was not sent by this wireless host.

  \item In about 4 seconds, the wireless host sends 15 802.11 Authentication
    frames to the AP.

  \item The host is attempting to use the Open System authentication algorithm,
    which does not require a key.

  \item The AP never replies to the wireless host's authentication requests.

  \item The wireless host tries to authenticate with the new AP at $t=63.16$s
    and receives a reply within one thousandth of a second.

  \item The 802.11 Association Request frame is made at $t=63.16$s.  The
    associated response is at $t=63.19$s.

  \item Thoe host is willing to transmit at 1Mbit/s, 2Mbit/s, 5.5Mbit/s,
    11Mbit/s, 6Mbit/s, 9Mbit/s, 12Mbit/s, and 18Mbit/s.  It can also support
    24Mbit/s, 36Mbit/s, 48Mbit/s, and 54Mbit/s.  The AP supports 1Mbit/s, 2Mbit/s, 5.5Mbit/s,
    and 11Mbit/s, as well as 6Mbit/s, 9Mbit/s, 12Mbit/s, 18Mbit/s, 24Mbit/s,
    36Mbit/s, 48Mbit/s, and 54Mbit/s.

  \item 802.11 Probe frames are sent when a host requests information about
    another host.  Responses contain capability information, including data
    rates.

\end{enumerate}
\end{document}
