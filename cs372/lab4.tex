\documentclass[11pt]{article}

\usepackage[top=1in,right=1in,left=1in,bottom=1in]{geometry}
\usepackage{hyperref}

\begin{document}

\begin{flushright}
  Kyle Cesare
\end{flushright}

{\center \textbf{Lab 4} \\}

\begin{enumerate}
  \item Ethernet address \texttt{8c:a9:82:97:87:f4}.

  \item Destination address: \texttt{00:25:00:fe:ba:9a}.  No, this is not
    \url{gaia.cs.umass.edu}.  This is simply the next hop in the route to the
    website.  In this case, it is my router.

  \item \texttt{0x0800}.  This corresponds to IP.

  \item The \texttt{G} is the 67th byte in the Ethernet frame.

  \item Source address: \texttt{00:25:00:fe:ba:9a}.  Again, this is not
    \url{gaia.cs.umass.edu}.  This is simply the next hop in the route to the
    website.  In this case, it is my router.

  \item Destination address \texttt{8c:a9:82:97:87:f4}.  This is the Ethernet
    address of my computer.

  \item \texttt{0x0800}.  This corresponds to IP.

  \item The \texttt{O} in \texttt{OK} is the 80th byte in the Ethernet frame.

{\tiny
\begin{verbatim}
Address                  HWtype  HWaddress           Flags Mask            Iface
10.0.1.1                 ether   00:25:00:fe:ba:9a   C                     wlo1
\end{verbatim}
}

  \item The \texttt{address} is the network layer address of the host.  In this
    case, it is an IP address.  The \texttt{HWtype} column specifies the type of
    connection to that host.  The \texttt{HWaddress} is the link-layer
    (Ethernet) address.  The \texttt{Iface} column is the Linux interface to
    which this entry is applicable.

  \item The source is my computer's address, \texttt{8c:a9:82:97:87:f4}.  The
    destination is \texttt{ff:ff:ff:ff:ff:ff}, which seems to be a broadcast
    address.

  \item The frame type is \texttt{0x0806}, which corresponds to the ARP
    protocol.

  \item
    \begin{enumerate}
      \item The \texttt{opcode} field begins after 144 bits, or 18 bytes.
      \item The payload of the \texttt{opcode} field can either be a
        \texttt{REQUEST} or a \texttt{REPLY}.
      \item The sender's IP address is stored in the \texttt{ar\$spa} field.
      \item The requested hardware address is given for the provided target
        protocol address, \texttt{ar\$tpa}.
    \end{enumerate}

  \item
    \begin{enumerate}
      \item The \texttt{opcode} field begins after 8 bytes.
      \item The \texttt{opcode} has a value of 2, which represents a
        \texttt{REPLY}.
      \item The answer to the initial request is available in the \texttt{Target
        MAC Address} field.
    \end{enumerate}

  \item The source is my router (\texttt{00:25:00:fe:ba:9a}) and the destination
    is my computer (\texttt{8c:a9:82:97:87:f4}).

  \item There is in fact an ARP reply, but it was not directed to us, so we can
    not view it.  As we saw, an ARP request broadcasts to all addresses, but the
    reply is only to the specific host that requested.  Therefore, the host
    requested for all addresses, then the ARP reply was sent only to that
    computer.

  \item If you entered the correct IP address but the wrong physical address,
    then all data would go to the wrong hardware location.  Likely, the provided
    \texttt{EtherAddr} will not exist, so the packets will be ignored.

  \item Timeouts are typically between 10 and 20 minutes.  This value was found
    at \url{http://www.tcpipguide.com/free/t_ARPCaching-2.htm}.

\end{enumerate}
\end{document}
