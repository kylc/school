\documentclass[12pt,letterpaper]{article}

\usepackage{fullpage}

\pagestyle{empty}

\begin{document}

{\center\textbf{Project 2 Write-up} \\}

\begin{enumerate}

\item The objective of this assignment is to become more familiar with both
  general kernel development strategies and procedures along with, specifically,
  gaining a better understanding of how Linux I/O schedulers are written.  As
  the previous assignment didn't involve writing any code by hand, this was my
  first exposure to writing kernel code.
  
\item For this project, we had to implement a new I/O scheduler for the Linux
  3.0.4 Kernel.  I approached this problem by first ensuring I had a good
  overview of Linux I/O scheduling by reading the chapters from \textit{Linux
  Kernel Development}.  I then dove into specific implementation details by
  reading through the source code of the CFQ, the deadline scheduler, and the
  NOOP scheduler.

  We based our design off of the NOOP scheduler, only changing the order in
  which requests were dispatched.  Upon dispatch, we first scan the queue
  looking for requests ahead of the previous request in the current direction.
  If there are any, we choose the closest one by sector count.  If there are
  none in the current direction, we scan in the opposite direction and, again,
  choose the closest.
  
  While this is an $O(n)$ operation and could probably be made more
  efficient, no performance constraints were placed on the assignment so we went
  with the simplest implementation.

\item To test our solution, we used the kernel's built-in \texttt{printk}
  function to print sector numbers along with the current elevator direction as
  we were dispatching requests.  We could then check the output of
  \texttt{dmesg} and verify that requests were being dispatched in the correct
  order.

\item Through this assignment, I gained a much better understanding of the
  kernel's build system.  In order to add an additional file we had to edit a
  sub-\texttt{Makefile} and add configuration options to the \texttt{Kconfig}.
  I was also able to play around with the \texttt{/sys} file system when setting
  the scheduler dynamically, and was able to see some of the other uses of the
  file system.

\end{enumerate}
\end{document}
