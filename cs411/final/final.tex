\documentclass[12pt,letterpaper]{article}

\usepackage{fullpage}
\usepackage{hyperref}
\usepackage{amsmath}

\begin{document}

{\center\Large\textbf{Analysis of the Android x86 Build Tree} \\}

\section{Introduction}

Android x86 is a project to port the Android operating system to the x86
architecture, which would allow it to run on more common x86 hardware. Here, I
will analyze the Android x86 kernel and compare and contrast the I/O scheduling
and process scheduling algorithms with those in the default kernel, and with
those on which we worked throughout the quarter.

\section{Kernel Design}

\subsection{I/O Scheduler}

Modern Linux kernels use the Completely Fair Queueing (CFQ) scheduler. This
scheduler attempts to fairly distribute I/O by keeping track of I/O time
per-process, and attempting to balance it with the process priority.

We developed a new I/O scheduler for the kernel based on the Shortest Seek Time
First (SSTF) algorithm. At each dispatch, The algorithm searches in the current
head direction for the nearest request. If it doesn't find a request, it
searches for the closest request in the opposite direction. We found that this
solution had significantly worse performance than the CFQ scheduler. While this
algorithm might provide great throughput when reading sequentially, it can
potentially ruin interactivity if a request is placed directly behind the drive
head.

According to the Android kernel documentation, no kernel options are
specifically set regarding the I/O scheduler. Therefore, at runtime, it will use
CFQ. This is a reasonable choice, as Android devices I/O loads should resemble
interactive system loads, which the CFQ scheduler was designed for.

With many of today's Android devices being powered by solid-state drives, our
elevator/seek time based solution would not work well. These devices do not add
additional overhead for random access requests, so the CPU power required to
order requests would be wasted. Additionally, for these devices latency is
probably more important than throughput. Our scheduler sacrifices latency in
favor of throughput, while the CFQ seems to maintain a good balance of the two.

\subsection{Process Scheduling}

The default scheduler in the Linux kernel is the Completely Fair Scheduler
(CFS). Because of the way the CFS is implemented, it still gives interactive
processes that spend a large amount of their time waiting a fair share of the
CPU.

We added functionality for first-in-first-out (FIFO) and round-robin (RR)
real-time scheduling. The FIFO algorithm places no timing constraints on a
process, allowing it to run indefinitely or until it yields or blocks.  When
this happens, it allows the next process, ordered by priority, to run.  The
round-robin scheduler does exactly the same, except it enforces a time slice.
If a program exhausts its time slice before yielding or blocking, then the
kernel preempts it and runs the next process.

The Android kernel does not differ from the mainline kernel as far as process
scheduling. It makes use of the Completely Fair Scheduler (CFS). Our FIFO and RR
schedulers would not have worked well in this situation. FIFO is designed for
batch systems, where a process runs to completion before yielding. This would
ruin interactivity for the system. The RR scheduler may be a little better due
to its enforcement of time slices, but it still meant for soft real-time
applications, which an interactive operating system is not.

The Android kernel does enable the \texttt{CONFIG\_NO\_HZ} configuration option,
which enables tickless mode. Typically, the kernel must interrupt anywhere
between 100 and 1000 times per second to schedule processes and perform general
kernel tasks. It must do this even if the system is idle. On a mobile device
with limited battery capacity, this consumes precious processing power and
wastes battery life. In a tickless system, when the processor is idle timer
ticks are disabled, which allows the system to go into a deep sleep mode.

The choice to use the default Linux scheduler along with the
\texttt{CONFIG\_NO\_HZ} option makes sense for Android. CFS has been shown to
have good performance in interactive systems with desktop-like workloads, which
is exactly where modern Android devices are headed. Little differentiates them
from modern desktop PCs, so there is little need for kernel changes.

\section{Group Evaluation}

\input{_group_eval.tex}

\begin{thebibliography}{9}

\bibitem{lkdev}
  \textit{Linux Kernel Development},
  Robert Love

\bibitem{androidx86}
  Android-x86 Project:
  \url{http://www.android-x86.org/}

\bibitem{ibmcfs}
  Inside the Linux 2.6 Completely Fair Scheduler:
  \url{http://www.ibm.com/developerworks/library/l-completely-fair-scheduler/}

\bibitem{androidkconfig}
  Android Kernel Configuration:
  \url{http://source.android.com/devices/tech/kernel.html}

\end{thebibliography}

\end{document}
