\documentclass[12pt,letterpaper]{article}

\usepackage{fullpage}

\pagestyle{empty}

\begin{document}

{\center\textbf{Project 1 Write-up} \\}

\begin{enumerate}

\item The objective of this assignment is to get familiar with our groups, set
  up our development environments, and get a cursory overview of kernel
  development.  This is facilitated through implementing two schedulers for the
  kernel: first-in-first-out and round-robin.

\item For this project, we had to implement two scheduling algorithms for the
  Linux 3.0.4 Kernel: round-robin scheduling and first-in-first-out scheduling.

  As an overview, the FIFO algorithm places no timing constraints on a process,
  allowing it to run indefinitely or until it yields or blocks.  When this
  happens, it allows the next process, ordered by priority, to run.  The
  round-robin scheduler does exactly the same, except it enforces a timeslice.
  If a program exhausts its timeslice before yielding or blocking, then the
  kernel preempts it and runs the next process.  Development is greatly
  simplified by realizing that the two algorithms are almost identical, the only
  difference being the addition timeslices in round-robin scheduling.  This
  means that they can both share a large amount of code.

\item To test our solution, we wrote a quick test application, attached to the
  bottom of this document.  The application uses \texttt{sched\_setscheduler} to
  tell the scheduler to treat it as either \texttt{SCHED\_FIFO} or
  \texttt{SCHED\_RR}.  The application has to run a CPU-bound task, so as to not
  block and be preempted by another task.  We chose to calculate a Fibonacci
  sequence.  We then observed the results of running the application.  In the
  \texttt{SCHED\_FIFO} mode we expected the entire operating system to stop
  responding during the run, as the processor would never be yielded.  In the
  \texttt{SCHED\_RR} setting, we expected more of an impact to overall system
  interactivity than with the default scheduler, but it would not lock up as
  with \texttt{SCHED\_FIFO}.

  Note that we had to ensure the virtual machine on which we were testing was
  restricted to only a single core.  Otherwise, the machine would still be able
  to run additional processes on other cores, and we may not lose interactivity
  as we were expecting.

\item This assignment was a great opportunity to explore the kernel's code
  structure at a high level.  Seeing how code was organized between device
  drivers, architecture-dependent code, kernel operations, etc. makes it
  immediately obvious how large an undertaking the kernel project is.  After
  delving into some of the source code, it was amazing to see some of the
  extremes to which the developers go to eek out the last bits of performance,
  using compiler hints (through macros) like \texttt{likely} and
  \texttt{unlikely} in conditionals, along with the pervasive use of function
  inlining.  Potentially the most helpful thing was just having everyone get a
  feel for the basics of kernel development workflow.  This involved SVN,
  building the kernel, installing the kernel, and booting the new kernel.

\newpage

\begin{verbatim}
#include <stdio.h>
#include <sys/types.h>
#include <unistd.h>
#include <sched.h>

#define MAX 100000000000

/* Modified from:
 * http://www.zacharyfox.com/blog/fibonacci-project/fibonacci-in-c
 */
int fibonacci(int n)
{
        int a = 0;
        int b = 1;
        int sum;
        int i;

        for (i = 0; i < n; i++) {
                sum = a + b;
                a = b;
                b = sum;
        }

        return 0;
}

int main(void)
{
        pid_t pid = getpid();

        struct sched_param param;
        param.sched_priority = 1;

        // int err = sched_setscheduler(pid, SCHED_FIFO, &param);
        int err = sched_setscheduler(pid, SCHED_RR, &param);
        printf("Error: %d\n", err);

        int f = fibonacci(MAX);
        printf("Fib: %d\n", f);

        return 0;
}
\end{verbatim}

\end{enumerate}
\end{document}
