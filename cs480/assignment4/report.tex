\documentclass[12pt,letterpaper]{article}

\usepackage{fullpage}

\pagestyle{empty}

\begin{document}

Kyle Cesare \hfill
\today \hfill

{\center\textbf{Milestone 4 Report} \\}

\begin{description}

\item[Specification] The purpose of this milestone is to begin implementation on
  the final stage of our compiler, the code generator. In this milestone, we do
  not implement loops are variables; we only allow constant expressions.
  
\item[Processing] I decided to base the design for the code generator on the
  object-oriented syntax tree I had already developed. I chose to add an
  \texttt{emit()} method on each syntax node type (e.g. if, while, print). Then,
  outputting valid GForth code would simply be a matter of post-order traversing
  the tree and building the output string.

  The next obstacle was handling int to float upcasting. In arthimetic
  operations involving both ints and floats, the ints must be moved to the
  floating point stack before the operation can occur. This is handled in the
  \texttt{Binop} emit code. It compares the types of the operands, and upcasts to
  floats if necessary.
  
  It is also able to warn if the input contains binops operating on types that
  don't match up, like if an input tried to compare a string and an int. It does
  not, however, fail on these inputs as there may be some valid reason for doing
  such a comparison (e.g. pointer arithmetic).
  
\item[Testing Requirement] To test for correctness, I tested the generator on a
  number of input samples. These included both samples that should and should
  not generate valid code, along with examples that shouldn't even parse or
  tokenize, to verify no regressions popped up in this milestone. Examples of
  programs that should not generate valid code are those that perform invalid
  operations on types, e.g. \texttt{5 S" Hello" +}. My compiler warns of these,
  but there is no way to easily correct them.
  
\item[Retrospective] This assignment was a culmination of what we have been
  working towards for the first part of the quarter. It ties together the
  concepts of the tokenizer and parser by attaching the code generation stage.
  The code generator also exposes any design flaws made in the design of the
  parser and the syntax tree.
  
\end{description}

\end{document}
