\documentclass[12pt,letterpaper]{article}

\usepackage{fullpage}

\pagestyle{empty}

\begin{document}

Kyle Cesare \hfill
\today \hfill

{\center\textbf{Milestone 3 Report} \\}

\begin{description}

\item[Specification] The purpose of this milestone is to develop a parser that
  reads tokens from the tokenizer written in milestone 2 and matches them to
  productions. It should transform the stream of tokens into a syntax tree.

\item[Processing] My first approach to this problem was to take some example
  programs and hand-derive them. This gave me an idea of what the recursive
  descent parser is supposed to do, and allowed me to start to form an
  algorithm. I then began translating the algorithm into code.

  I had to make some revision to my tokenizer to allow it to be called in the
  way the parser needs, and to allow for a single peek token. With these
  changes, however, it was easy to integrate it into the parser.

  The parser starts in the start production, then recursively descends into
  matching productions based on the peek token. Once it reduces to terminals, it
  returns a syntax node and the recursion reaches its base. From here, each
  production is able to form a level in the tree, where its recursive steps are
  its children.
  
\item[Testing Requirement] To test for correctness, I tested the parser on a
  number of input samples. These included both samples that should and should
  not parse, along with examples that shouldn't even tokenize, to verify no
  regressions popped up in this milestone.
  
\item[Retrospective] I learned how to take a formal grammar from definition to
  implementation. I also learned the significance of an LL(1) grammar, and what
  makes a grammar parsable with only a single lookahead from the tokenizer. I
  also gained an understanding of how to remove left recursion from a grammar, a
  necessary step to writing a predictive parser.
  
\end{description}

\end{document}
