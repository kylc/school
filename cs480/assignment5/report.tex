\documentclass[12pt,letterpaper]{article}

\usepackage{fullpage}

\pagestyle{empty}

\begin{document}

Kyle Cesare \hfill
\today \hfill

{\center\textbf{Milestone 5 Report} \\}

\begin{description}

\item[Specification] This final milestone wraps up the front-end of the
  compiler. My compiler now supports all syntax and semantics as defined by
  IBTL. This milestone adds support for \texttt{let}, \texttt{assign} and
  \texttt{while} statements.
  
\item[Processing] Because of my OOP design from last milestone, this milestone
  was comparatively easy to implement. The main task was simply adding
  \texttt{let}, \texttt{assign}, and \texttt{varaccess} nodes to my tree.
  Additionally, I needed to store the variables and their types somewhere.

  My design of the variable map allows for intuitive scoping. Every node in the
  tree has its own variable map. Let nodes add the variable to their
  parent's variable map, so as to make the variables available to any other
  statements at the same level or deeper. When a variable is looked up, it
  checks the current map, then continues up the tree until the variable is found
  or the root is reached. In the second case, this means that a variable was not
  defined and an error is thrown.

  Because of this design, a variable access node simply has to lookup the
  variable name in its tree, and the hierarchy will automatically be handled.
  Any variables declared higher in the tree will be found and used, whereas any
  variables lower in the tree will not be used as it is out of scope.

  My implementation of local variables uses GForth's stack-based locals. These
  are defined by \texttt{\{ a b c \}}, where each variable pops the value on top
  of the stack. Therefore, when variables are defined, the compiler first pushes
  a bunch of default values to the stack. These variables may also have types to
  specialize their stacks via the \texttt{F: a S: b} syntax.

  For IBTL's while loops, I use GForth's \texttt{BEGIN ... UNTIL} syntax. In
  this syntax, the top of the stack at the end of each iteration is evaluated to
  true or false to determine if the loop should continue. Given that IBTL
  defines a while loop but this is an until loop, the flag must be inverted
  before evaluation with the \texttt{invert} word.

  In this milestone, I also handled the ambiguous case of the \texttt{-}
  operator. The compiler no longer tokenizes this with the float/int token.
  Instead, it always tokenizes it as a binop. Then, when emitted, the compiler
  checks how many child nodes the operator has. It can then decide if the minus
  is a binop or a unop, and handles it using either GForth's \texttt{-} or
  \texttt{negate}.

  To wrap up the design, I've started using Boost Log for output. This allows be
  to log items at various levels of severity. Then, I can set a filter to
  declare what minimum severity I want outputted. This makes it very easy to
  implement the compiler's normal output mode, in which only the resulting
  GForth code is emitted, and the verbose mode, in which a number of debugging
  statements are emitted throughout the compilation process.

  A note about warnings and possible errors: If a possible type mismatch is
  detected, either in binops or variable assignments, a warning is issued. This
  warning will not be printed unless the compiler is run in verbose mode. This
  is to avoid a situation where the program is doing something tricky (but
  correct), causing the compiler to fail outputting any code. To see warnings,
  run the compiler with the \texttt{-v} option.
  
\item[Testing Requirement] To test this milestone, I ran the compiler on a
  number of inputs and fed the output into GForth directly. Test cases included
  string concatenation, if statements, basic arithmetic with variables, loops
  with variables, and a rectangular integration program.

  % TODO: Scoping?
  
\item[Retrospective] Now that this compiler project has finished, it is easy to
  see the connection to the theory learned in class to the application of
  writing compilers.

  It is also clear how important it is to make good design decisions from the
  beginning. I don't think it was uncommon for students to rewrite their
  tokenizers and parsers three or four times throughout this project. However,
  even choosing a good design isn't good enough. The final module may need
  support for some unforeseen feature that simply can't be accomplished using
  the current design, and so the module will need to be rebuilt. This was one
  frustrating aspect of this class: while implementing one module of the
  compiler, we didn't yet know enough about the requirements of the module that
  would be using it make many of these design choices.
\end{description}

\end{document}
