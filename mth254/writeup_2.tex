\documentclass[11pt]{article}

\usepackage[top=1in,right=1in,left=1in,bottom=1in]{geometry}

\usepackage{amssymb}
\usepackage{amsmath}

\begin{document}

\begin{flushright}
  Kyle Cesare
\end{flushright}

{\center \textbf{Math Write-up 2} \\}

Similar to the way a parallelogram can be considered a rectangle when finding
its area, a parallelepiped can be considered a rectangular prism.  To find the
volume, we will simply use $V = Ah$, where $A$ is the area of the base and $h$
is the height.

\vspace{10mm}

First, let's solve for the area of the base.

$$
  A = |\vec v||\vec w|
$$

Now, let's find the height.

$$
  h = |\vec u|\cos\theta
$$

To find the total volume of the parallelepiped, multiple these together.

$$
  V = Ah = |\vec v||\vec w||\vec u|\cos\theta
$$

Now, let's compare this with what we are given.

$$
  V = |\vec u \cdot (\vec v \times \vec w)| = | |\vec u|(|\vec v||\vec w|\sin\phi)\cos\theta |
$$

However, in all parallelepipeds, the angle between $\vec v$ and $\vec w$ will be
$\frac{\pi}{2}$, so $\sin\phi$ will always be $1$.  Therefore, we can remove it.

$$
  V = | |\vec u|(|\vec v||\vec w|)\cos\theta | = |\vec v||\vec w||\vec u|\cos\theta
$$

This is exactly the same as what we derived earlier.

\end{document}
