\documentclass[11pt]{article}

\usepackage[top=1in,right=1in,left=1in,bottom=1in]{geometry}

\usepackage{amssymb}
\usepackage{amsmath}

\begin{document}

\begin{flushright}
  Kyle Cesare
\end{flushright}

{\center \textbf{Math Write-up 7} \\}

Given the plane $x + 3y + z = 6$, maximize the volume of the solid formed
between the plane and the xy-axis on the rectangle $R$ with the verticies $(0,
0)$, $(a, 0)$, $(0, b)$, and $(a, b)$, where $a + 3b = 6$.

\vspace{8mm}

To solve this problem, we must first find a formula that represents the volume
of the solid.  We will find this by tasking the double integral of the plane.
Then, we will maximize the volume on the given domain.

\vspace{2mm}

First, we must find the volume by taking the double integral.

$$
  R = \{ (x, y) : 0 \le x \le a, 0 \le y \le b \}
$$

\begin{align*}
  \int_0^b \int_0^a 6 - x - 3y \text{ dxdy}
    &= \int_0^b \left[ 6 - \frac{1}{2}x^2 - 3xy \right]_0^a \text{ dy} \\
    &= \int_0^b 6a - \frac{1}{2}a^2 - 3ay \text{ dy} \\
    &= \left[ 6ay - \frac{1}{2}a^2y - \frac{3}{2}ay^2 \right]_0^b \\
    &= -\frac{1}{2}ab \left[ a+3(b-4) \right]
\end{align*}

Now we need to find the maximum of this function on the domain $a + 3b = 6$.
The easiest way to do this is to simply make a table of some possible values of
$a$ and $b$, and finding their values.  Using this method, we find that the
maximum is at the point $(3, 1)$ with a value of $9$ cubic units.

\vspace{0.8mm}

\begin{center}
\begin{tabular}{ | c | c | }
  \hline
  $(x, y)$           & $f(x, y)$ \\
  \hline
  $(0, 2)$           & $0$ \\
  $(1, \frac{5}{3})$ & $5$ \\
  $(2, \frac{4}{3})$ & $8$ \\
  $(3, 1)$           & $9$ \\
  $(4, \frac{2}{3})$ & $8$ \\
  $(5, \frac{1}{3})$ & $5$ \\
  $(6, 0)$           & $0$ \\
  \hline
\end{tabular}
\end{center}

\end{document}
