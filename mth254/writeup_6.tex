\documentclass[11pt]{article}

\usepackage[top=1in,right=1in,left=1in,bottom=1in]{geometry}

\usepackage{amssymb}
\usepackage{amsmath}

\begin{document}

\begin{flushright}
  Kyle Cesare
\end{flushright}

{\center \textbf{Math Write-up 6} \\}

Given the equation of a plane $x - y + z = 2$, find the point on the plane
closest to the point $P(1, 1, 1)$.

\vspace{8mm}

To solve this problem, we must find an equation for the distance between a point
$(x, y, z)$ on the plane, then minimize it.  The resulting point is our answer.

\vspace{2mm}

First, we know that the distance between points in 3-dimensional space is
$d=\sqrt{x^2 + y^2 + z^2}$.  Therefore, the distance between a point $(x, y, z)$
on our plane and the point $P(1, 1, 1)$ is

$$
  d(x, y, z) = \sqrt{(x-1)^2 + (y-1)^2 + (z-1)^2}
$$

Now, find the partial derivatives in the $x$ and $y$ directions.  Fortunately,
$d^2$ will have the same critical points as $d$, so we can simply find the
derivatives of $d^2$ rather than having to worry about the square root.

\begin{align*}
  f(x, y) = (d(x, y, z))^2 &= (x-1)^2 + (y-1)^2 + (z-1)^2 \\
                           &= x^2 - 2x + 1 + y^2 - 2y + 1 + z^2 -2z + 1
\end{align*}

From the original equation of the plane, we know that $z = 2 + y - x$.  We can
substitute that back in to simplify the equation.

\begin{align*}
  f(x, y) &= x^2 + y^2 + (2 + y - z)^2 - 2x - 2y - 2(2 + y - x) + 3 \\
          &= 2x^2 + 2y^2 - 2xy - 4x + 3
\end{align*}

Now let's take the partial derivatives

$$
  \frac{\partial f}{\partial x} = 4x - 2y - 4 \text{ and }
  \frac{\partial f}{\partial y} = 4y - 2x
$$

Set these equations to zero and we find that the solution is $x = \frac{4}{3}$
and $y = \frac{2}{3}$.  Plugging those back into the original equation, we also
get that $z = \frac{4}{3}$.  We can verify that this is a minimum with the
Second Derivative Test.

\begin{align*}
  \frac{\partial^2 f}{\partial x^2} = 4 \text{ and }
  \frac{\partial^2 f}{\partial y^2} = 4 \text{ and }
  \frac{\partial^2 f}{\partial y \partial x} = -2
\end{align*}

Using the second derivative test $f_{xx}(x, y)f_{yy}(x, y) - (f_{xy}(x, y))^2$,
we find that the second derivative test yields a positive result at all points
along the curve.  Since $f_{xx}$ is positive, we verify that the point is a
local minimum.  As the surface rises to infinity in all directions around our
plane, we can determine that this is the absolute minimum.

\end{document}
